\chapter{Tog: Language and Type Checker}
\label{ch:tog}

%The interpreter of our DSL generates definitions of theories and related constructions. 
To implement the methodology we presented in Chapter~\ref{ch:design}, we need a language for representing and manipulating theories and a type checker to verify these manipulations. 
Theories are written in some formal language, the object language. To manipulate them we need to investigate and manipulate the syntax of the object language. This can be done in the same language if it has a strong reflection mechanism, or in the meta language in which the object language is embedded. As the main goal of our work is to investigate the usefulness of a generative approach, we do not want to be constrained by the amount of support given by the reflection mechanism. Working in the meta language gives us full control over manipulating the object language's syntax. 
We need our meta language to support the following features in the object language it represents:
\begin{itemize}
\item $\Pi$-Types: The semantics of the combinators we are using is given in categorical dependent logic. Having $\Pi$-types is needed to represent the types of views in terms of their source and target theories. 
\item Dependent records to represent theories as telescopes. 
\item A module system to manage namespaces such that every theory with its generated constructions is a module. 
\item Inductive data types to represent term languages. 
\item Equality to represent the equations within a theory. 
\end{itemize}

These features are available in most dependently typed systems, like Agda, Coq, and Lean. But we refrained from using any of these systems to avoid delving into their design decisions. Instead, we prefer a small language that does not have many other extra features. We use Tog~\cite{tog}, a small implementation of Martin-L\"{o}f type theory. 
It provides a small dependently typed language and type checker. It was created by the Agda developers to experiment with type checking ideas. It has mainly been used to experiment with type checking through unification~\cite{mazzoli2016type}. 

Tog is implemented in Haskell. Figure~\ref{fig:togRepr} shows its internal representation. 
% , style=trac
\begin{figure}
\begin{minted}[fontsize=\scriptsize,escapeinside=~~,mathescape=true]{haskell}
data Decl
    = TypeSig TypeSig
    | FunDef Name [Pattern] FunDefBody
    | Data Name Params DataBody
    | Record Name Params RecordBody
    | Module_ Module
    | ~$\cdots$~
    deriving (Eq, Ord, Show, Read)

data TypeSig = Sig Name Expr
    deriving (Eq, Ord, Show, Read)

data Where = Where [Decl] | NoWhere
    deriving (Eq, Ord, Show, Read)

data Params
    = NoParams | ParamDecl [Binding] | ParamDef [HiddenName]
    deriving (Eq, Ord, Show, Read)

data HiddenName = NotHidden Name | Hidden Name
    deriving (Eq, Ord, Show, Read)

data DataBody
    = DataDecl Name | DataDef [Constr] | DataDeclDef Name [Constr]
    deriving (Eq, Ord, Show, Read)

data RecordBody
    = RecordDecl Name
    | RecordDef Name Fields
    | RecordDeclDef Name Name Fields
    deriving (Eq, Ord, Show, Read)

data Fields = NoFields | Fields [Constr]
    deriving (Eq, Ord, Show, Read)

data Constr = Constr Name Expr
    deriving (Eq, Ord, Show, Read)

data FunDefBody = FunDefNoBody | FunDefBody Expr Where
    deriving (Eq, Ord, Show, Read)

data Telescope = Tel [Binding]
    deriving (Eq, Ord, Show, Read)

data Binding = Bind [Arg] Expr | HBind [Arg] Expr
    deriving (Eq, Ord, Show, Read)

data Expr
    = Lam [Name] Expr
    | Pi Telescope Expr    -- $\Pi$ types 
    | Fun Expr Expr        -- function types  
    | Eq Expr Expr         -- equations 
    | App [Arg]            -- type applications 
    | Id QName             -- types names
    deriving (Eq, Ord, Show, Read)

data Arg = HArg Expr | Arg Expr
    deriving (Eq, Ord, Show, Read)

data Pattern
    = EmptyP Empty | ConP QName [Pattern] | IdP QName | HideP Pattern
    deriving (Eq, Ord, Show, Read)
\end{minted}
\caption{Internal Representation of the Tog Language}
\label{fig:togRepr}
\end{figure}

A Tog module is a list of declarations, such that each declaration is either a type signature, function definition, datatype declaration, record definition, or a nested module represented using the  \lstmath{TypeSig}, \lstmath{FunDef}, \lstmath{Data}, \lstmath{Record}, and \lstmath{Module$\_$} constructors, respectively. 
According to the type \lstmath{Decl}, modules can import and open other modules, but our experience shows that this feature is not supported. 

Parameters to modules, records, and datatypes are represented by the \lstmath{Params} type. A single parameter has type \lstmath{Binding} and can be declared implicit by using the constructor \lstmath{HBind}. 

A record field and a datatype constructor are both of type \lstmath{Constr}, each having a name and a type expression \lstmath{Expr}. Dependent types are created with the \lstmath{Pi} constructor. Function types are curried and represented with the \lstmath{Fun} constructor. Axioms that are equations are represented with \lstmath{Eq} constructor. Type and function applications are created using the \lstmath{App} constructor. If a type has no arguments, then the \lstmath{Id} constructor is used to refer to it by name. 

To perform pattern matching, the \lstmath{Pattern} type is used. Matching with a $0$-ary constructor is done using \lstmath{IdP}. If the constructor takes parameters, then \lstmath{ConP} is used. \lstmath{HideP} represents pattern matching on implicit arguments and \lstmath{EmptyP} represents the don't care \lstmath{$\_$} character. 

We extend Tog to support the input theory expressions and flatten them into Tog dependent records and morphisms between them. 
The structure of these dependent records is used to generate new constructions. The generated constructions can be records, datatypes, or functions presented in Tog syntax. The well-typedness of the generated constructions is ensured by the Tog type checker. 

%The fact that tog has no universes (Set : Set). that we couldn't find a better experimental language, and that this would not affect the generated constructions because we do not perform any reasoning and our code is backed by universal algebra, so we have a meta theory that tell us that we are not generating nonsense 
%\ednote{We probably need to say something about universes in Tog. In tog, it is true that \lstmath{Set : Set}. We decided then that this is not a big deal, but I don't know why.}


\chapter{Library}
\label{ch:library}
%One can define a new theory either by stating its components or by using combinators to build it from existing ones. 

In this Chapter, we build a library of axiomatic theories representing the algebraic hierarchy, before we use it to generate related constructions that we discuss in Chapter~\ref{ch:generation}. 

Our library consists of equational first-order theories organized as a theory graph using the tiny theories approach. Instead of having to provide all declaration of the theories and morphisms within the graph, we use the MathScheme combinators introduced in~\cite{CaretteOConnorTPC, carette2018building}. 

It is common to see the algebraic hierarchy as a series of inclusions as in Figure~\ref{fig:flatExtensions}. 
\begin{figure}
\centering{
\begin{tikzcd}
\cn{Magma} \arrow[r, hook] & \cn{Semigroup}\arrow[r, hook] & \cn{Monoid} \arrow[r, hook] & \cn{Group} \arrow[r, hook] & \cdots
\end{tikzcd}}
\caption{Algebraic structures as extensions}
\label{fig:flatExtensions}
\end{figure}  
But the algebraic hierarchy is more rich than that, considering for example the list in~\cite{jipsen}. 
In Section~\ref{sec:background:tinytheories} we discuss tiny theories as an adequate approach to capturing this structure. We present the example of building the theory of \lstmath{Unital} by extending the theory of \lstmath{PointedMagma} to create \lstmath{LeftUnital} and \lstmath{RightUnital}, then combining them. This example is described by a diamond structure as in Figure~\ref{fig:unitalDiamond}. 
\begin{figure}
\centering{
\begin{tikzcd} 
& \verb|LeftUnital| \arrow[dr,hook] & \\
\verb|PointedMagma| \arrow[ur,hook] \arrow [dr,hook] & & \verb|Unital| \\
& \verb|RightUnital| \arrow[ur,hook] &
\end{tikzcd}}
\caption{The diamond in the definition of \lstmath{Unital}}
\label{fig:unitalDiamond}
\end{figure}
The diamond structure appearing in the definition of \lstmath{Unital} is not a special case. Instead, diamonds are pervasive in the algebraic hierarchy, as shown in the theory graph for defining \lstmath{Monoid} in Figure~\ref{fig:cube_monoid}. 
%examining the work in \cite{halleck} and \cite{jipsen} show us that the algebraic %hierarchy is more packed with diamonds, as we show in Figure~


%\begin{figure}[h]
	\begin{tikzcd}[row sep=huge, column sep=scriptsize]
		&& & \verb|Pointed0| \arrow[dd,hook] & \\
		\verb|Carrier| \arrow[dd,hook] \arrow[rr,hook] & & \verb|Pointed| \arrow[ur,mapsto] & & \\ 
		& \verb|AddMagma| \arrow[rr,hook] \arrow[dd,hook]& & \verb|AddPointedMagma| 
		\arrow[rr,hook] 
		\arrow[dd,hook]& & 
		\verb|AddRightUnital|  \arrow[dd,hook]\\
		\verb|Magma| \arrow[ur,mapsto] \arrow[dd,hook]  & & 
		\verb|PointedMagma| \arrow[dd,hook] \arrow[ur,mapsto] \arrow[from=uu, crossing over] 
		\arrow[rr,hook,crossing over] \arrow[from=ll,hook, crossing over]
		& & \verb|RightUnital| \arrow[ur,mapsto] \\ 
		& \verb|AddSemigroup| \arrow[ddd,hook] & &\verb|AddLeftUnital| \arrow[rr,hook] &  & 
		\verb|AddUnital| \arrow[dddllll,hook] \\
		\verb|Semigroup|  \arrow[ddd,hook] \arrow[ur,mapsto]&  & \verb|LeftUnital| 
		\arrow[ur,mapsto] 
		\arrow[rr,hook,crossing over] &  & \verb|Unital| \arrow[dddllll,hook,crossing 
		over]\arrow[ur,mapsto] 
		\arrow[from=uu,hook,crossing over]& \\ 
		&&&& \\ 
		&  \verb|AddMonoid| &  &&\\ 
	 \verb|Monoid|  \arrow[ur,mapsto] && &&
	\end{tikzcd}
	\caption{Defining Monoid using tiny theories}
	\label{fig:cube_monoid}
%\end{figure}	
.
%As we discussed in Section~\ref{sec:thry_graph_in_action}, dealing with diamonds is a challenging problem~\cite{sakkinen1989disciplined, jigsaw1992, traits2006, diamonds2011}. There is a gap in how specification systems handle them, whether by not following its correct semantics, or by not supporting it at all. 

But the diamond structure does not come without problems. We need to have careful infrastructure to deal with them in order to avoid the diamond problem~\cite{jigsaw1992,traits2006,diamonds2011}, a.k.a multiple inheritance or the fork-join problem~\cite{sakkinen1989disciplined}. 

In Section~\ref{sec:thry_graph_in_action} we provide an overview of the support for morphisms in different formal systems. Section~\ref{sec:msCombinators} introduces the MathScheme combinators which creates a morphism-based approach to building theory graphs, leading to a solution for the diamond problem. We discuss our implementation of the combinators in Section~\ref{sec:lib_implementation}. Using them to build the library is discussed in Section~\ref{sec:guidelines}. We end up with a discussion of best practice for using the combinators in Section~\ref{sec:library:discussion}. 

\begin{comment}
To test our generation algorithms, we needed a large library of equational theories. As we have discussed in Section~\ref{sec:broader_context}, we work in the favor of a library organized as a theory graph, believing that it leverages the structure of mathematical knowledge. Arrows of the graph are the means to relating the different theories. In this section, we present our approach to building a library that emphasizes these connections. 

In Section~\ref{sec:thry_based_libs} we discuss the motivation behind building such a library. In Section~\ref{sec:ms_combinators} we present the combinators used in building it and discuss how they are arrow based. Section~\ref{sec:lib_implementation}, discusses the challenges of the implementation of the combinators to build a theory graph. We finally show some interesting cases of library definitions in Section~\ref{sec:interesting_cases}. 
\end{comment}


\section{Theory Graph Development}
\label{sec:thry_graph_in_action}

Although many formal systems support theory graph structure, more support to using and defining arrows is needed. 
Specware and MMT force users to provide all details of theories and morphisms between them. IMPS, in some cases, generate morphisms given source and target theories. 

%On the other hand, it leads to a multiple diamonds (a.k.a. multiple inheritance) in the development of the hierarchy. 
%Many formal systems support a theory graph approach, and realizes the need of having morphisms between theories. Examples of these are Clear, OBJ, CASL, Maude, Specware, IMPS, and MMT. 

%Clear is - to our knowledge - the first system that provides a modular way to write formal specifications. It provide theory combinators to build larger theories from smaller ones. 
%\ednote{A good resource for Clear is the PhD thesis of Sanella with the title: semantics, implementation and pragmatics of CLEAR} 
Another way to support building a library rich in morphisms is to provide combinators to handle some of the work. Clear is - to our knowledge - the first system to use combinators for creating new theories\footnote{Clear is a specification language, and theories are used under the name specifications.}. OBJ and CASL are successors of Clear that also support combinators. We focus our discussion on CASL as a representative to these systems, as it is the only living one now and so we were only able to look at its library and run experiments on it. 
We realize two problems related to combinators in CASL. First, It is not always possible to flatten theories built through the use of combinators, mainly when using hiding and freeness combinators~\cite{CoFI:2004:CASL-RM}. The second problem is related to how the \emph{union} operation is implemented. The \lstmath{union} operator is the one that handles multiple inheritance. Although the semantics of union is a pushout in the category of specifications and morphisms, it is computed on a 'same name, same thing' basis~\cite{bidoit2003casl}. Figure~\ref{fig:casl_expr} shows the problems that occur from using this principle. 
\begin{figure}
\begin{tikzpicture}[node distance=2cm]
    \node(NodeName1){
\begin{tcolorbox}[colback=white, width= 0.45\textwidth,left=0pt,right=0pt,top=0pt,bottom=0pt]
\begin{caslcode}
spec BaseSpec = sort A end 
spec Ext1 = BaseSpec then 
 ops e  :A 
   __*__:A * A -> A,unit e 
end 
spec Ext2 = BaseSpec then 
 ops e :A 
   __+__:A * A -> A, unit e 
end 
spec Combine = 
   Ext1 and Ext2 
end
\end{caslcode}   
\end{tcolorbox}
    };
\node(NodeName2) [right=of NodeName1] {  % added position of second box
    \begin{tcolorbox}[colback=white, width= 0.45\textwidth,left=0pt,right=0pt,top=0pt,bottom=0pt]
\begin{caslcode}
  sorts A
  op __*__ : A * A -> A
  op __+__ : A * A -> A
  op e : A
  forall x : A . x + e = x 
   %(ga_right_unit___+__)%
  forall x : A . e + x = x 
   %(ga_left_unit___+__)%
  forall x : A . x * e = x 
   %(ga_right_unit___*__)%
  forall x : A . e * x = x 
   %(ga_left_unit___*__)%
\end{caslcode} 
        \end{tcolorbox}
        };
\draw[black, line width=2mm,-{Triangle[angle=60:1pt 3]}] (NodeName1) -- (NodeName2);
\end{tikzpicture}
\caption{CASL union operation: On the left, the specification \lstinline|Combine| is defined as the union of \lstinline|Ext1| and \lstinline|Ext2|. On the right, the declarations of specification 
\lstinline|Combine| as computed by CASL.}
\label{fig:casl_expr}
\end{figure}
Both specifications \verb|Ext1| and \verb|Ext2|, on the left side, extend the \verb|BaseSpec| with a binary operation and its unit element. The semantics of the \lstmath{combine} (or \lstmath{and}) combinator is presented as a pushout in the category of specifications. A pushout between the two arrows \lstmath{BaseSpec $\;\to\;$ Ext1} and \lstmath{BaseSpec $\;\to\;$ Ext2} would result in a theory with one sort \lstmath{A},and  two binary operations with two different unit elements. When trying this specification in CASL\footnote{using the online tool at: \url{http://rest.hets.eu}}, it computes the declarations on the right side of the figure which has only one unit element for the two binary operations. This is different from what a pushout would compute. 

We performed the same experiment with Isabelle locale expressions~\cite{ballarin2003locales} and got similar results. In the following section, we introduce a collection of combinators that provide solid infrastructure for dealing with theories and morphisms to build a large library organized as a theory graph. 

\section{MathScheme Combinators}
\label{sec:msCombinators}

Combinators provide algebra for manipulating theories in different ways. They enhance modularity, reusability and maintainability of the library by saving the user the need to repeat definitions. \cite{carette2018building} introduces $4$ combinators based on the definitions of theories as contexts, and theory morphisms in dependent type theory as we discuss them in Sections~\ref{sec:background:theory} and \ref{sec:background:morphisms}. %The combinators form a language to create a librarnew theories by reusing older ones. We use this language to build a library of $250$ theories as we discuss in Section~\ref{sec:lib_implementation}. 
%many theorem provers that are in use today, do not have the notion of morphism and suffer from redundancy. Agda and Coq are big examples of that. Apart from extensions, they do not support combining modules.

%In~\cite{carette2018building}, we present $4$ combinators along with their operational and categorical semantics. The semantics we provide is based on the category of contexts and the categorical semantics of dependent type theory. In the following section we present the combinators and their semantics. 

%Despite the large literature on using combinators to save work and reduce redundancy, Most of the systems we listed here suffer from redundancies in their own libraries. \ednote{Add an appendix about the different redundancies}To avoid this redundancy, we suggest in~\cite{carette2018building} a set of combinators that form a compact language to describe algebraic theories by reusing older ones. We use these combinators to build our library as well as some important design decisions 
A library built using these combinators respect the following design decisions 
\begin{itemize}
    \item Theories can always be flattened. Not all users of a formal system are interested in the hierarchy used to build the theories they need. A mathematician who wants to prove results in \verb|Group| theory is only interested in groups with their standard definitions and results. This user should not be forced to work with groups as extensions of some theory, like \verb|Monoid|. Abstracting over the hierarchy in users' code also has the advantage that the code need not change in case the hierarchy change, like in the case of changing the type class hierarchy in Haskell~\cite{wiki:haskell_hierarch}. 
    \item Names are taken seriously. Similar concepts have different names in different contexts of mathematics. The unit of \verb|_+_| has a different name than the one of \verb|_*_| and confusing their names would be a huge usability problem. The combinators introduced in~\cite{carette2018building} does not generate any names nor attempt to use any heuristics to solve name clashes. Instead name clashes are detected and the library developer is asked to resolve them.  
    \item Tiny theories are systematically used. Since we do not provide a drop combinator, we use tiny theories to make sure all intermediate results are available for future theories to use. 
    \item Arrows are the main building unit of the library. The semantics and the implementation of the combinators is based on arrows, not theories. This makes it possible compute category theory operations based on their real semantics, not an approximation like same-name-same-thing. 
\end{itemize}


The combinators assume the underlying logic in which theories are defined to be a dependent type theory (DTT). Therefore, a theory is viewed as a context, or a telescope as defined in equation~\ref{eq:telescope}. But a specific variant of DTT is not assumed, instead many of the details are abstracted from. The minimum requirements of the underlying DTT are listed in ~\cite{carette2018building}. We include them here for coherence. These requirements are 
\begin{itemize}
    \item An infinite set of variable names \vars.
    
    \item A typing judgement for terms $s$ of type $\sigma$ in a context
    $\Gamma$ which we write $\Gamma \vdash s : \sigma$.
    
    \item A kinding judgement for types $\sigma$ of kind $\kappa$ in a context
    $\context{\Gamma}$ which we write\\
    $\context{\Gamma} \vdash \sigma : \kappa : \Box$.  We further assume that the set
    of valid kinds $\kappa : \Box$ is given and fixed.
    
    \item A definitional equality (a.k.a. convertibility) judgement of terms
    $s_1$ of type $\sigma_1$ and $s_2$ of type $\sigma_2$ in a context $\context{\Gamma}$,
    which we write $\context{\Gamma} \vdash s_1 : \sigma_1 \equiv s_2 : \sigma_2$. \ We
    will write $\context{\Gamma} \vdash s_1 \equiv s_2 : \sigma$ to denote $\context{\Gamma} \vdash
    s_1 : \sigma \equiv s_2 : \sigma$.
    
    \item A notion of substitution on terms. Given a list of variable
    assignments $\assignment{x_i}{s_i}{i < n}$
    and an expression $e$ we write $\substitutiondef{e}{x_i}{s_i}{i < n}$
    for the term $e$ after simultaneous substitution of variables $\left\{ x_i
    \right\}_{i < n}$ by the corresponding term in the assignment.
\end{itemize}

We now introduce the combinators we use from~\cite{carette2018building} 
\subsection{Extension} 
\label{subsec:extension}
Extensions are the most basic combinator. On its own, it makes it possible to define a flat hierarchy as in Figure~\ref{fig:flatExtensions}. 

The input to an extension combinator are a theory presentation $\Gamma$ and a list of declarations $\extDecls = \left\{a_{i}:\sigma_{i}:\kappa_{i}\right\}_{i<n}$. 
The combinator computes a new theory (\lstmath{pres}) $\Gamma\rtimes\Delta^+$ and an identity morphism (\lstmath{$\tilde{\text{id}}$}) from $\Gamma$ to $\Gamma\rtimes\Delta^+$, where $\rtimes$ is asymmetric operation that adds definitions to a telescope. On one side $\Gamma$ is a well-formed theory, but $\Delta^+$ may not be well-formed on its own. 
The construction is defined as
\[\extensionDef{\extSource}{\extDecls}\]
An extension is well-formed if 
\begin{eqnarray}
\forall i \cdot a_i \notin \syms{\Gamma_{i-1}} \\
\forall i \cdot \Gamma_{i-1} \vdash \sigma_i : \kappa_{i}
\end{eqnarray}
where $\Gamma_{i-1} = \Gamma \rtimes \{a_0 : \sigma_0 : \kappa_0\  \cdots \ a_{i-1} : \sigma_{i-1} : \kappa_{i-1}\}$ 

%\paragraph{Categorical Semantics}
%In the category of contexts $\ctxcat$, an extension corresponds to a forgetful functor. 

\paragraph{Example}
Extensions are used when new concepts are added. According to little theories, the concept need to be added in its smallest context, i.e. if $\Gamma \vdash c : t$ then for every $\Sigma \subset \Gamma$, $\Sigma \nvdash c : t$. Tiny theories encourages adding one new declaration at a time. A good example is adding properties of a binary operation, like \lstmath{commutativity} or \lstmath{associativity} as follows\footnote{The syntax we use here is the one used in our implementation. We give brief explanations for them here, and introduce them in details in the next section.} 
\begin{togcode}
Semigroup = extend Magma {assoc : { x y z : A } ~$\to$~ op x (op y z) ==
                                                   op (op x y) z } 
CommMagma = extend Magma {comm  : { x y : A } ~$\to$~ op x y == op y x }
\end{togcode} 
where \lstmath{Magma} is the theory $\Gamma$ being extended, \lstmath{assoc} and {comm} are definitions in $\Delta^+$. 

\subsection{Rename}
\label{subsec:rename}
Renames do not add new concepts, but allows using flexible notations while still reusing all results from the source theory. 
Given a theory presentation $\Gamma$ and a rename function $\pi$, the output of the rename operation is a new theory (\lstmath{pres}) $\pi \cdot \Gamma$ which is computed by performing a substitution of $\pi$ into the declarations of $\Gamma$, and an embedding morphism $\tilde{\pi} : \Gamma \to \pi\cdot\Gamma$ that maps symbols of $\Gamma$ to those of $\pi\cdot\Gamma$ based on the renaming function $\pi$. 
\[ \renameDef{\renSource}{\renFun} \]

A rename operation is well-formed whenever the rename function $\pi : |\Gamma| \to \vars$ is an injection, and the codomain is a $k-$permuation on $\vars$, where $k$ is the number of declarations in $\Gamma$. 

\paragraph{Example}
After defining \lstmath{Semigroup} in the example of the previous section over a binary operation \lstmath{op}, one would want to define the additive and multiplicative versions using the symbols \lstmath{+} and \lstmath{*}, resp. It also make sense to have a morphism from \lstmath{Semigroup} to those variants that only differ in the names of the symbols. The rename combinator does just that 
\begin{togcode}
AddSemigroup  = rename Semigroup {op to +} 
MultSemigroup = rename Semigroup {op to *} 
\end{togcode}  

\subsection{Combine}
\label{subsec:combine}
Conisder the following small library
\begin{togcode} 
Theory Empty = {} 
Carrier = extend Empty {A : Set}
Pointed = extend Carrier {e : A}
Magma   = extend Carrier {op : A -> A -> A}
\end{togcode} 
The flattened version of the theories of these libraries are 
\begin{itemize}
\item[] \lstmath{Carrier = [A : Set]}
\item[] \lstmath{Magma = [A : Set, op : A $\;\to\;$ A $\;\to\;$ A]} 
\item[] \lstmath{Pointed = [A : Set, e : A]} 
\end{itemize}
Now we want to define the theory \lstmath{PointedMagma} which has a binary operation and a point. It makes sense to assume this theory to be an extension of both \lstmath{Magma} and \lstmath{Pointed}. Using the extension combinator will not help us here. In this situation, we are facing the diamond problem in which our new theory is inheriting from two theories and it is not clear whether a declaration, lilke \lstmath{A : Set} should be repeated or not. 
The situation is more complicated if we consider the definition of \lstmath{AdditiveSemigroup} by relating it to \lstmath{AdditiveMagma} defined as 
\begin{togcode} 
AdditiveMagma = rename Magma {op to +} 
\end{togcode} 
and \lstmath{Semigroup} defined as in Section~\ref{subsec:extension}. Here we have the same binary operation with different names. Which name should be used? Or should they be repeated, having two  binary operations in the outcome? 

The case when a theory needs to be related to more than one is prevalent when building large libraries. As we see in these examples, it occurs very early on when formalizing the algebraic hierarchy. The combine operation supports the multiple inheritance situation by relying on the information in the morphisms. Combine performs a pushout of the morphisms in the category of theory presentations, i.e. a pullback in the category of contexts.  A pushout is a $5-$ary operation that takes $2$ arrows and $3$ objects of a category. The arrows need to originate from the same source. The $3$ theories can be deduced from the arrows as the two target theories of the arrows and their common source. For cases where there are name clashes, like the name clash between \lstmath{op} and \lstmath{+} in the \lstmath{AdditiveSemigroup} example, the user is required to provide renames to resolve it. This is consistent with our design decision to not use heuristics or name generation to resolve any name conflicts. 

The two morphisms of the combine operation are $u_{\Delta}$ and $u_{\Phi}$ both having $\Gamma$ as their source, and having $\Delta$ and $\Phi$, resp, as their targets. 
\[
\comfun\left( u_{\Delta}, u_{\Phi}, \pi_{\Delta}, \pi_{\Phi}\right) \define
\left\{\begin{aligned}
\mathtt{pres} & = \combineResult_0\rtimes\left(\combineResult_{\Delta} \cup \combineResult_{\Phi}\right) \\
\mathtt{embed}_{\Delta} & = \left[v_{\Delta}\right] : \Delta\rightarrow\combineResult\\
\mathtt{embed}_{\Phi} & = \left[v_{\Phi}\right] : \Phi\rightarrow\combineResult\\
\mathtt{diag} & = \left[uv\right]:\Gamma\rightarrow\combineResult\\
\end{aligned}\right\}\]
% ----------------- the removed mediate ----------------- 
% \mathtt{mediate} & = \lambda\ w_{\Delta}\ w_{\Phi}\ .\  w_{\combineResult}
% --------------------------------------------------------
 
%To say a \lstmath{Monoid} is a \lstmath{Semigroup} and a \lstmath{Unital}, we need multiple inheritance, which creates a diamond structure. The \lstmath{Combine} operation is the one to use in this case. 
%It is well agreed upon that the semantics of such an operation should be a pushout in the category of theory presentations\ednote{cite the other systems or refer to a previous section}.
$\pi_{\Delta}$ and $\pi_{\Phi}$ are two rename functions given to resolve name conflicts. 

There is the following precondition for performing a combine 
\begin{equation}
\pi_{\Delta} \left( x \right) = \pi_{\Phi} \left( y \right)
\Leftrightarrow \exists z \in \left| \Gamma \right|.\ x =
\substitution{z}{u_{\Delta}}{} \wedge y = \substitution{z}{u_{\Phi}}{} 
\label{eq:combinePrecond}
\end{equation}
for every \lstmath{x $\;\in$ $\mid\Delta\mid$} and \lstmath{y $\;\in$ $\mid\Phi\mid$}. 
This precondition ensures that after applying the rename function on symbols \lstmath{x} and \lstmath{y}, if they map to the same symbol then they have originated from the same symbol in the source theory $\Gamma$. It also ensures that there are no name clashes when mapping a symbol $z$ across the two morphisms and rename functions.  

\paragraph{Example}
We have given two examples in the beginning of this section illustrating situations in which combine operations is needed. A \lstmath{PointedMagma} is defined as 
\begin{togcode} 
PointedMagma = combine Magma {} Pointed {} 
\end{togcode} 
The embeddings being combined are \lstmath{Carrier $\;\hookrightarrow\;$ Magma} and \lstmath{Carrier $\;\hookrightarrow\;$ Pointed}. The empty 
\lstinline|{}| means the identity rename functions are used in this expression, as in this case no name clashes need to be resolved. 

The \lstmath{AddSemigroup} is defined as 
\begin{togcode} 
AdditiveSemigroup = combine AdditiveMagma {} Semigroup {op to +} 
\end{togcode} 
The embeddings used here are \lstmath{Magma $\;\mapsto\;$ AdditiveMagma} and \lstmath{Magma $\;\hookrightarrow\;$ Semigroup}. The declaration \lstmath{op} in \lstmath{Magma} is mapped to \lstmath{+} in \lstmath{AdditiveMagma} and remains as \lstmath{op} in \lstmath{Semigroup}. Therefore, a rename \lstinline|{op to +}| is needed to resolve this name clash. 
%The work in \cite{carette2018building} defines $4$ combinators used to build a graph of axiomatic theories in a dependently-typed logic. This graph corresponds to a diagram in the category $\tpcat$ of theory presentations and morphisms between them.
%    \item The category of contexts $\ctxcat$ is the opposite of the category of theory presentations $\tpcatOp$, i.e. if a theory \lstmath{T'} is the extension of a theory \lstmath{T} in the category $\tpcat$, then there is a morphism in $\ctxcat$ from \lstmath{T'} to \lstmath{T} that drops some of the declarations in \lstmath{T}. Because theories are viewed as telescopes, to drop a declaration, all its subsequent ones need to be dropped. 
 %   \item A fibered category is \ednote{short description here}. The category of context $\ctxcat$ forms the basis of a fibration, with the fibered category being the category of extensions $\extcat$. The fibration maps every extension object from $\extcat$ to its source in $\ctxcat$. 

\section{The Flattener}
\label{sec:lib_implementation}
The combinators from~\cite{carette2018building} has been implemented in~\cite{cicm2019diagrams, TPCProto, meta-prim-blog}. 
Experiments have been performed on applying them for creating a large library of algebraic constructions~\cite{mathscheme2011experiments}. With the exception of~\cite{cicm2019diagrams}, the other implementations and the associated library did not emphasize the morphisms in the way presented in~\cite{carette2018building} and summarized in the previous section. Instead, they based their computation of pushout on theories using same-name-same-thing approach, which makes errors like the one in Figure~\ref{fig:casl_expr} go undetected.  
This approach also computes results for expressions that should not be meaningful in the language of combinators presented in~\cite{carette2018building}. Consider the following expression
\begin{togcode}
SemiRng = combine AdditiveCommutativeMonoid Semigroup Ringoid
          over RingoidSig
\end{togcode}
Note that the definition of \lstmath{SemiRng} above is different from the syntax we presented before. First, there is no mention of the rename function, as they all are the identity. Second, there are $3$ theories to be combined, which means $3$ arrows. Normally, we talk about combining $2$ arrows with the semantics being the pushout. In this case, we have $3$ arrows each has the source \lstmath{RingoidSig} and the three targets are the arguments to combine. In this case, a colimit is needed, not a pushout. Finally, there is the \lstmath{over} part of the expression which was part of the original work in~\cite{CaretteOConnorTPC} and then removed in the more recent version~\cite{carette2018building}. The over part specifies what the source of the morphisms is. In Section~\ref{subsec:overPart} we discuss why we need an \lstmath{over} part in the combine declaration. 

An implementation that reflects the principles of the combinators will not be able to find a morphism between \lstmath{RingoidSig} (the common source) and \lstmath{Semigroup} (the second target) to compute the expression above. 
The theory \lstmath{RingoidSig} has declarations for two binary operations, while \lstmath{Semigroup} has only one. A morphism from \lstmath{RingoidSig} to \lstmath{Semigroup} needs to drop one binary operation. This is not possible given the choice of combinators that avoids a drop operation. 

It is worth noting that by implementing the combinators we mean computing a flattened version of the theory presentation described by the given expressions. This is performed by the flattener that given a theory presentation produces a tog dependent record of declarations within the described theory presentation. 

In Section~\ref{subsec:coreDecisions} we discuss a modification in the syntax of combine and why we have it. We introduce the syntax of the language we implement in Section~\ref{sec:impl:expressions} and start discussing the implementation in Section~\ref{subsec:theoriesMorphisms} by presenting how we represent theories and morphisms in our framework. In Section~\ref{subsec:graph} we present the type of the theory graph. The implementation of the combinators that build the graph is presented in Section~\ref{subsec:combinatorsImpl}. 

\subsection{Referring to arrows}
\label{subsec:overPart}
%\label{subsec:coreDecisions}
%We slightly modify the restrict the original combinators to solve some implementation issues. We conclude by summarizing the syntax of the tog expressions that we use based on the syntax of the combinators. 
%\subsubsection{Referring to arrows}
The extension and rename combinators needs to identify a theory in the graph to operate on and compute the output theory and arrow. The input theory is part of the expression of the combinator. 
In the case of \lstmath{combine}, the inputs to the combinator are two morphisms and two rename functions. But the syntax for combine is not defined in terms of arrows. Instead, it is defined in terms of theories and the arrows are left to the implementation to infer them. 
For example, the expression 
\begin{togcode}
combine CommutativeMagma {} AssociativeMagma {}
\end{togcode}
does give information that the targets of the two embeddings involved are \lstmath{CommutativeMagma} and \lstmath{AssociativeMagma}, but it does not specify the source of the embeddings. The algorithm has three choices of the source theory, which is common to both morphisms. 
\begin{itemize}
    \item If the source theory is \lstmath{Magma}, the theory resulting from the combine operation will have one binary operation that is both associative and commutative 
    \item If the source theory is \lstmath{Carrier}, then the definition is describing a theory (along with the related morphisms) that has two binary operations, one associative and the other commutative. But this theory will not be computed because of the name clash; The user has to choose another name for one of the two operations. A possible fix is 
    \begin{togcode}
combine CommutativeMagma {op to +} AssociativeMagma {op to *}
    \end{togcode}
\end{itemize}
As the hierarchy gets deeper this problem becomes more complicated. For example, \lstmath{CommutativeGroup} and \lstmath{AssociativeGroup} have many more possibilities for their common source. 

The reason of this problem is that the syntax of the language presented in~\cite{carette2018building} is based on naming target theories, while the syntax is based on having the embedding available. This leaves the gap of using the target theories to infer the embeddings. Using theories, instead of morphisms, in the syntax is a usability decision. Arrows do not have canonical names, mainly because they do not appear in informal mathematics. For example, it is hard to think of a name for the arrow (the result of composition of arrows) from the \lstmath{PointedMagma} to \lstmath{Monoid} theory. It is easier to refer to it in terms of the source and the target than to give it any name. 

We use an approach that still uses theories for usability reasons but gives more information for inferring the embeddings. We modify the syntax of \lstmath{combine} in the paper to have an \lstmath{over} part similar to the initial work on the combinators~\cite{CaretteOConnorTPC}. 

\begin{comment}
\subsubsection{All Paths Commute Approach} 
\ednote{@JC: Do we need to remove mixin, so all paths commute?} 
When referring to an arrow using its source and target, we implicitly assume that all paths commute, i.e.: Given the source and target, they either is no path, one path, or multiple paths that commute between them. 

In Section~\ref{sec:background:morphisms}, we discussed the three types of arrows, identity, embeddings and general morphisms. We also noted in Section~\ref{subsec:mixin} that the only combinator that accepts and generates a general morphism is \lstmath{mixin}. If we restrict our language to \lstmath{extension}, \lstmath{rename}, and \lstmath{combine}, we end up with an all-embeddings graph, in which all paths commute. 
\end{comment}

\subsection{Theory Expressions}
\label{sec:impl:expressions}
The language that we implement has the following syntax
\begin{togcode}
 Map m = {a~$_0$~ to b~$_0$~ ; ~$\cdots$~ ; a~$_n$~ to b~$_n$~}
 Theory T = {a~$_0$~ : t~$_0$~ ~$\cdots$~ a~$_n$~ : t~$_n$~}
 extend T {a~$_0$~ : t~$_0$~ ~$\cdots$~ a~$_n$~ : t~$_n$~}
 rename T m 
 combine T~$_1$~ m~$_1$~ T~$_2$~ m~$_2$~ over T
 id from T~$_1$~ to T~$_2$~
\end{togcode} 
%{a~$_0$~ to b~$_0$~ ; ~$\cdots$~ ; a~$_n$~ to b~$_n$~}
%{a'~$_0$~ to b'~$_0$~ ; ~$\cdots$~ ; a'~$_n$~ to b'~$_n$~}
where \lstmath{T}, \lstmath{T$_1$}, and \lstmath{T$_2$} are theories, \lstmath{m}, \lstmath{m$_1$}, and \lstmath{m$_2$} are mappings that are either previously defined using the \lstmath{Map} keyword or expanded as a list of mappings 
\lstmath{$\{$a$_0$ to b$_0$ ; ... ; a$_n$ to b$_n$ $\}$}. 

Although one can declare a theory with a list of declarations using the \lstmath{Theory} keyword, we only use it to create the empty theory. 


\subsection{Theories and Morphisms}
\label{subsec:theoriesMorphisms}
We defined a theory in DTT in Section~\ref{sec:background:theory} as a telescope.  It is captured by the type \lstmath{GTheory}. 
\begin{hscode}
data GTheory = GTheory {
  declarations :: [Constr],
  waist        :: Int     }
\end{hscode}
The waist is needed to determine how many of the declarations are parameters, as in~\cite{alhassy2019}. 

The type \lstmath{GView} describe the morphism as defined in~\ref{sec:background:morphisms}. 
\begin{hscode}
data GView  = GView {
  source  :: GTheory,
  target  :: GTheory,
  rename :: Rename }  
\end{hscode}
where \lstmath{type Rename = Map.Map Name_ Name_} is the mapping function. 

\subsection{Theory Graph Structure}
\label{subsec:graph}
We define a theory graph as a name-to-theory map and a name-to-view map 
\begin{hscode}
data TGraph = TGraph { 
  _nodes :: Map.Map Name_ GTheory,
  _edges :: Map.Map Name_ GView } 
\end{hscode}
An alternative way to represent graphs would have been to include only the \lstmath{_edges}, as they contain information about theories. We preferred to keep both mappings to make it easier to lookup theories in the graph. 

We noticed that in many cases, the same renames are being reused. So, we also added a \lstmath{Mapping} type that allows the user to define something like 
\begin{hscode}
Map plus-zero = {op to + ; e to 0}
\end{hscode}
and reuse it. Accordingly, a library consists of a theory graph and some mappings. 
\begin{hscode}
data Library = Library {
  _graph   :: TGraph,
  _renames :: Map.Map Name_ Rename }
\end{hscode}

\subsection{Combinators}
\label{subsec:combinatorsImpl}
The language extension that we introduce to tog is described in the type \lstmath{Language} 
\begin{togcode}
data Language = 
    MappingC Name [RenPair]
  | TheoryC Name [Constr]
  | ModExprC Name ModExpr
\end{togcode}
where \lstmath{MappingC} creates a mapping function, \lstmath{TheoryC} creates a theory from a list of declarations, and \lstmath{ModExprC} is the constructor for creating theory expressions. We discuss them in the following sections. 

%We introduced the expressions we use to build the theory graph in Section~\ref{sec:impl:expressions}. In this section, we discuss our implementation for these expressions. The language that we introduce to Tog is described here 

\subsubsection{Mappings}
A definition of a mapping is elaborated into an entry in the \lstmath{renames} list of the library. 
\begin{hscode}
addMapping :: Name -> [RenPair] -> Library -> Library
addMapping nm rens = 
   over mappings (Map.insert (nm^.name) (renPairsToMapping rens))
\end{hscode}
\lstmath{over} is the setter function we get by using Haskell lenses. It sets the \lstmath{mappings} field of the library to a new instance of \lstmath{Map} that adds the new mapping to the ones in the input library. 

\subsubsection{Flat Theories}
Given a theory presentation as a list of declarations, we construct the new theory and add it to the list of theories in the graph without any morphisms connecting them to other theories. 
\begin{hscode}
theory :: Name -> [Abs.Constr] -> Library -> Library
theory nm cList =
  let newThry  = GTheory cList waistNm
  in  over graph (over nodes (Map.insert (nm^.name) newThry))
\end{hscode}


\subsubsection{Theory Expressions}
The syntac for the theory expression is introduced in Section~\ref{subsec:coreDecisions}. We now discuss their implementation. 
We start with the function \lstmath{updateGraph} which adds theories and morphisms to the graph, assuming they already have been computed by the combinator
\begin{hscode}
updateGraph ::   Name_ -> Either GView PushOut -> TGraph -> TGraph
updateGraph nm (Left view) =
  over nodes (Map.insert nm (target view)) .
  over edges (Map.insert ("To"++nm) view)
updateGraph nm (Right ut) =
  over nodes (Map.insert nm (target ~$\$$~ uLeft ut)) .
  over edges (\e -> foldr (uncurry Map.insert) e 
                        [("To"++nm++"1",uLeft ut),
                         ("To"++nm++"2",uRight ut),
                         ("To"++nm++"D",diagonal ut)])
\end{hscode}
The first argument to \lstmath{updateGraph} is the name of the new theory. Then, the function expects the arrows resulting from the combinator to be added to the graph. We know that all the combinators compute only one new theory. 
But, the number of computed arrows is different based on the combinators. \lstmath{extends} and \lstmath{rename} generates one morphism, while \lstmath{combine} generates three. We capture this with the type \lstmath{Either GView PushOut}, where \lstmath{Pushout} is defined as 
\begin{hscode}
data PushOut = PushOut { -- of a span
  uLeft    :: GView,
  uRight   :: GView,
  diagonal :: GView,
  apex     :: GTheory } -- common point
\end{hscode}
The names of the new arrows are generated based on the names of the new theories. Since a new theory with a user given name is defined every time, we know that the new arrow names have not been generated before. 

The functions \lstmath{computeExtend}, \lstmath{computeRename}, and \lstmath{computeCombine} calculate the new arrows and theories. 

\paragraph{1. Computing Extension}
The inputs to the extension operation is the theory being extended and the new declarations. The new theory is obtained by concatenating the new declarations to the ones already in the theory, given that there is no name clashes between new constructs, and that they are well-typed in the context presented by the theory declarations. 

The resulting view has the input theory as source, the computed theory as target. The identity mapping is computed using the \lstmath{validateRen} function, which assigns a mapping to every symbol in the input theory. In the case of extension the mapping is the identity. 
\begin{hscode}
computeExtend :: [Constr] -> GTheory -> GView
computeExtend newDecls srcThry =
  GView srcThry (extThry newDecls srcThry) (validateRen srcThry Map.empty)

extThry :: [Constr] -> GTheory -> GTheory 
extThry newConstrs thry@(GTheory constrs wst) =
  if List.intersect newConstrNames (symbols thry) == []
  then GTheory (constrs ++ newConstrs) wst
  else error ~$\$$~ "Cannot create theory "
    where newConstrNames = map getConstrName newConstrs
\end{hscode}

\paragraph{2. Computing Rename}
Computing renames requires computing substitutions. This requires traversing the internal representation of the theory and performing substitution as needed. We use Haskell's scrap-your-boilerplate package~\cite{syb}, based on~\cite{syb2003Jones}, to perform the traversal. The substitution is then performed using the \lstmath{gmap} function. 
\begin{hscode} 
gmap :: (Generics.Typeable a, Generics.Data b) => (a -> a) -> b -> b
gmap r x = Generics.everywhere (Generics.mkT r) x  
\end{hscode} 
\lstmath{gmap} traverses an instance of type \lstmath{b} changing every instance of \lstmath{a} according to the input function \lstmath{r}. \lstmath{computeRename} uses \lstmath{gmap} to perform substitution to declarations of the input theory, as follows
\begin{hscode}
computeRename :: Rename -> GTheory -> GView  
computeRename namesMap thry =
  GView thry (renameThy thry namesMap) (validateRen thry namesMap)

renameThy :: GTheory -> Rename -> GTheory
renameThy (GTheory constrs wst) m =
  GTheory (gmap (mapAsFunc m) constrs) wst
\end{hscode}

\paragraph{3. Computing Combine}
The algorithm to compute the result of combining two embeddings work as follows 
\begin{itemize}
    \item Given the name of the source theory and the two theories to be combined, the first step is to lookup the paths from the source to the target theory. A path is defined as a non-empty list of views. The function \lstmath{getPath} searches the graph for a path between given source and target theories. It starts at the target node and goes backwards, exploring the possible paths until it finds the source. Because none of the combinators result in backward arrows, we know the theory graph has no cycles, it actually looks more like a tree. Therefore, this simple search for a path algorithm works. The two paths are used to construct two instances of \lstmath{QPath}. 
\begin{hscode}
data QPath = QPath { 
  path :: Path,
  ren  :: Rename }
\end{hscode}

    \item At this point we have the two embedding and the two rename functions. The next step is to check the preconditions of \lstmath{combine} as in equation~\ref{eq:combinePrecond}. The function \lstmath{checkGuards} checks that all symbols in the source theory are mapped to the same symbol after applying the rename function. The scope checker of tog ensures the background direction of the equivalence in equation~\ref{eq:combinePrecond}. If the two instances of \lstmath{QPath} passes the precondition, the pushout can be computed.
    
    \item The result theory is computed by taking the disjoint union of the declarations in the source theory, the one on the left of the diamond (the first argument), then the one on the right. Note that this operation is not commutative. If we take the disjoint union of the source, right, then left theories, we get an equivalent but not equal theory. The order of declarations will be different, but the two theories will have the same declarations. 
\begin{hscode}
 newThry = 
   GTheory (disjointUnion3 (declarations srcMapped)
                           (declarations lThry) 
                           (declarations rThry)) 
           (waist srcMapped)
\end{hscode}    
    \item The source and target of the resulting morphisms are easy to figure out. The function \lstmath{allMaps} calculate the mappings by composing the mappings in the views on the path between the two theories, and then the one described by the rename function. 
\begin{hscode}
lView = GView lt newThry ~$\$$~ validateRen lt (allMaps left)
rView = GView rt newThry ~$\$$~ validateRen rt (allMaps right)
diag  = GView commonSrc newThry ~$\$$~ validateRen commonSrc (allMaps left)
\end{hscode}
\end{itemize}

\section{Library Building}
\label{sec:guidelines}
Once implementation of combinators is in place, we start using them to build the library organized as a theory graph. Our guide in building this library are the definitions in~\cite{msLibDecls}, which were part of an experiment~\cite{mathscheme2011experiments} on the way to develping the combinators we discuss in this chapter. Therefore, there are some definitions in that library that referred to non-existing arrows, like the definition of \lstmath{SemiRng} presented in Section~\ref{sec:lib_implementation}. As the implementation of combinators depend on finding the right arrows in the underlying theory graph, we had to work out the correct morphisms. We build a library of $227$ theories describing the algebraic hierarchy using tiny theories approach. Those theories range from \lstmath{Empty} up to \lstmath{Ring} and  \lstmath{BoundedDistributedLattice}.
 
The examples we give in Section~\ref{sec:msCombinators} gives an intuition of how the combinators work together to build the library. In this section we discuss some the challenges we faced to build the graph defining \lstmath{AdditiveMonoid} as in Figure~\ref{fig:cube_monoid}.  
%leading to a completely morphism-based development of the library. 

\subsection{Defining AdditivePointedMagma}
The very first theories of the algebraic hierarchy is defined as 
\begin{togcode} 
Carrier = extend Empty {A : Set}
Pointed = extend Carrier {e : A}
Pointed0 = rename Pointed {e to 0} 
Magma = extend Carrier {op : A -> A -> A}
AdditiveMagma = rename Magma {op to +} 
PointedMagma = combine Pointed {} Magma {} over Carrier
\end{togcode} 
These definitions would result in the black theories and arrows in Figure~\ref{fig:addPointedMagma}. Now we want to defined \lstmath{AdditivePointedMagma} consisting of three declarations \lstmath{(A,+,0)} such that all the blue arrows of 
Figure~\ref{fig:addPointedMagma} are generated. 
\begin{figure}[h]
    \begin{tikzcd}
   \verb|Empty| \arrow[r,hook] & \verb|Carrier| \arrow[r,hook] \arrow[d,hook] \arrow[rd,hook] & \verb|Magma| \arrow[r,mapsto] \arrow[d,hook] & \verb|AdditiveMagma| \arrow[dd,hook,dashed,blue] \\
  & \verb|Pointed| \arrow[d,mapsto] \arrow[r,hook] & \verb|PointedMagma| \arrow[rd,mapsto,dashed,blue]&    \\
   &  \verb|Pointed0|  \arrow[rr,hook,dashed,blue] & & 
   \textcolor{blue}{\cn{AdditivePointedMagma}} \\       
    \end{tikzcd}
    \caption{The construction of AdditivePointedMagma}
    \label{fig:addPointedMagma}
\end{figure}
Using one \lstmath{combine} to define it, we end up with the one of the following cases. 
\begin{itemize}
 \item  
\begin{togcode}
combine AdditiveMagma {} Pointed0 {} over Carrier
\end{togcode}
would generate the theory \lstmath{AdditivePointedMagma}, and the three arrows 
\begin{itemize}
    \item \lstmath{Carrier $\;\to\;$ AdditivePointedMagma}, 
    \item \lstmath{AdditiveMagma $\;\to\;$ AdditivePointedMagma}, 
    \item \lstmath{Pointed0 $\;\to\;$ AdditivePointedMagma}. 
\end{itemize}
The arrow \lstmath{PointedMagma $\;\to\;$ AdditivePointedMagma} won't be generated. 

\item  
\begin{togcode}
combine AdditiveMagma {} PointedMagma {op to +} over Magma 
\end{togcode}
will not generate the arrow \lstmath{Pointed0 $\;\to\;$ AdditivePointedMagma}. 

\item 
\begin{togcode}
combine Pointed0 {} PointedMagma {e to 0} over Pointed 
\end{togcode}
will not generate the arrow \lstmath{AdditiveMagma $\;\to\;$ AdditivePointedMagma}. 
\end{itemize}
Instead, to get all these connections, we define \lstmath{AdditivePointedMagma} as follows 
\begin{togcode}
Pointed0Magma = 
  combine Pointed0 {} PointedMagma {e to 0} over Pointed
PointedPlusMagma = 
  combine AdditiveMagma {} PointedMagma {op to +} over Magma
AdditivePointedMagma = 
  combine Pointed0Magma {op to +} PointedPlusMagma {e to 0} 
  over PointedMagma
\end{togcode}
which result in the graph in Figure~\ref{fig:addPointedMagmaReal}.  
\begin{figure}[h]
    \begin{tikzcd}
        Carrier \arrow[r,hook] \arrow[d,hook] \arrow[rd,hook] & Magma \arrow[r,mapsto] \arrow[d,hook] \arrow[rd,mapsto,dashed,blue] & AdditiveMagma \arrow[d,hook,dashed,blue] \\
        Pointed \arrow[d,mapsto] \arrow[r,hook] \arrow[rd,mapsto,dashed,blue]& PointedMagma \arrow[rd,mapsto,dashed,blue] \arrow[d,mapsto,dashed,blue] \arrow[r,mapsto,dashed,blue] &  PointedPlusMagma \arrow[d,hook,dashed,blue] \\
        Pointed0  \arrow[r,hook,dashed,blue] & Pointed0Magma \arrow[r,hook,dashed,blue] & \textcolor{blue}{AdditivePointedMagma} \\       
    \end{tikzcd}
    \caption{The construction of AdditivePointedMagma}
    \label{fig:addPointedMagmaReal}
\end{figure}
Although it is not immediately obvious to define \lstmath{AdditivePointedMagma} this way, it corresponds more to the tiny theories approach that advocates to having all intermediate theories. The two intermediate theories \lstmath{Pointed0Magma} and \lstmath{PointedPlusMagma} becomes useful when we define the \lstmath{Zero0} theory, which is defined as follows 
\begin{togcode}
PointedTimesZeroMagma = 
  combine PointedTimesMagma zero Pointed0Magma times 
  over PointedMagma 
Zero0 = 
  combine Zero times-zero PointedTimesZeroMagma {} 
  over PointedMagma 
\end{togcode}

\subsection{Defining AdditiveMonoid}
One would want to have \lstmath{AdditiveMonoid} with all the morphisms we introduced in Figure~\ref{fig:cube_monoid}. %But how close we can get to this diagram. 
We have discussed the construction of \lstmath{AdditivePointedMagma}, and showed how the construction is not precisely as in Figure~\ref{fig:cube_monoid}, although all the morphisms are defined, some are composed of other morphisms. Now, we focus more on the part of defining \lstmath{AdditiveUnital}. 

The definition of \lstmath{AdditiveLeftUnital} and \lstmath{AdditiveRightUnital} goes as follows 
\begin{togcode}
AdditiveLeftUnital = 
  combine AdditivePointedMagma {} LeftUnital plus-zero 
  over PointedMagma
AdditiveRightUnital = 
  combine AdditivePointedMagma {} RightUnital plus-zero 
  over PointedMagma
\end{togcode} 
It make sense to expect \lstmath{AdditiveUnital} to have morphisms with \lstmath{Unital}, \lstmath{AdditiveLeftUnital}, and \lstmath{AdditiveRightUnital}. Similar to the case we had in the previous section, a pushout will only compute two of these three morphisms. The possible pushouts are 
\begin{togcode}
combine AdditiveLeftUnital {} AdditiveRightUnital {} 
over AdditivePointedMagma
\end{togcode}

\begin{togcode}
combine AdditiveLeftUnital {} Unital plus-zero 
over LeftUnital 
\end{togcode}

\begin{togcode}
combine AdditiveRightUnital {} Unital plus-zero 
over RightUnital 
\end{togcode}

Unlike the one before, there is no way to compose different pushouts to get the three morphisms. The reason is that all possible pushouts compute the same theory, without the system realizing they are the same. 

The same problem occurs when defining \lstmath{AdditiveMonoid} and attempting to generate the three morphisms 
\begin{itemize}
    \item \lstmath{AdditiveUnital $\;\longrightarrow\;$ AdditiveMonoid}
    \item \lstmath{AdditiveSemigroup $\;\longrightarrow\;$ AdditiveMonoid}
    \item \lstmath{Monoid $\;\longrightarrow\;$ AdditiveMonoid}
\end{itemize}

We considered the possibility of using colimits or diagram combinators as in~\cite{cicm2019diagrams}. In either case, we want to arrive at the right pushouts and build diagrams or colimits on top of it. The solution we adopted here is to enable the user to add identity embedding between theories. This allowed us to substitute for arrows that are not generated in situations like the one we describe here. 
%Our approach is to define these techniques in terms of pushouts, i.e. provide them as a syntactic sugar to performing multiple pushouts. Since pushouts in the category of contexts are associative, we believed we are able to do that, but the examples above shows that this is not always possible. Therefore, we provided a construction that allows the user to add identity arrows between two theories.\ednote{need to add guarantees that this does not mess with the all-commute-path principle.} 

The declarations that we use to define \lstmath{AdditiveMonoid} are 
\begin{minted}[style=trac, fontsize=\footnotesize]{haskell} 
Theory Empty = {} 
Carrier = extend Empty {A : Set}
Pointed = extend Carrier {e : A}
Pointed0 = rename Pointed {e to 0} 
Magma = extend Carrier {op : A -> A -> A}
AdditiveMagma = rename Magma {op to +} 

Pointed0Magma = 
  combine Pointed0 {} PointedMagma {e to 0} over Pointed
PointedPlusMagma = 
  combine AdditiveMagma {} PointedMagma {op to +} over Magma
AdditivePointedMagma = 
  combine Pointed0Magma {op to +} PointedPlusMagma {e to 0} 
  over PointedMagma

Semigroup = extend Magma {associative_op : {x y z : A} -> 
                          op (op x y) z == op x (op y z) }
AdditiveSemigroup = combine AdditiveMagma {} Semigroup plus over Magma

LeftUnital = extend PointedMagma { lunit_e : {x : A} -> op e x == x }
RightUnital = extend PointedMagma { runit_e : {x : A} -> op x e == x }

AdditiveLeftUnital = combine AdditivePointedMagma {} 
                             LeftUnital plus-zero over PointedMagma 
AdditiveRightUnital = combine AdditivePointedMagma {} 
                              RightUnital plus-zero over PointedMagma 

Unital = combine LeftUnital {} RightUnital {} over PointedMagma
AdditiveUnital = combine AdditivePointedMagma {} Unital plus-zero over PointedMagma
idUnital = id from AdditiveRightUnital to AdditiveUnital 

Monoid = combine Unital {} Semigroup {} over Magma
AdditiveMonoid = combine AdditiveUnital {} Monoid plus-zero over Unital 
idMonoid = id from AdditiveSemigroup to AdditiveMonoid   
\end{minted} 

\section{Discussion}
\label{sec:library:discussion}
In many cases, there are many ways to define a theory. We restrict using \lstmath{extension} to adding new concepts within their minimal context, like adding associativity to \lstmath{Magma}. Whenver associativity is needed in a different context, it should be transported through \lstmath{rename} and \lstmath{combine}. In other words, a concept should only be defined once and transported to different theories via morphisms. It is also reasonable to assume that \lstmath{AdditiveMagma} should be an ancestor for any theory that contain the binary operation \lstmath{+}. This means that many renames take place using \lstmath{combine} operation, rather than the \lstmath{rename} one. For example, compare the following teo definitions of \lstmath{AdditiveSemigroup} 
\begin{togcode}
1. AdditiveSemigroup = rename Semigroup {op to +}
\end{togcode}
\begin{togcode}
2. AdditiveSemigroup = 
       combine AdditiveMagma {} Semigroup {op to +} over Magma 
\end{togcode}
Definition $1$ connects \lstmath{AdditiveSemigroup} only to \lstmath{Semigroup}, but definition $2$ creates more embeddings and connects it to \lstmath{AdditiveMagma}, \lstmath{Semigroup} and \lstmath{Magma}, which enriches the graph with useful morphisms. 

We also find that using theories that are deeper in the hierarchy when possible makes adds more structure for the graph. For example, here are two possible definitions of \lstmath{CommutativeGroup}   
\begin{togcode} 
1. CommutativeGroup = 
       combine CommutativeMagma {} Group {} over Magma
\end{togcode} 
\begin{togcode}
2. CommutativeGroup = 
       combine CommutativeMonoid {} Group {} over Monoid
\end{togcode}
The first definition does not connect \lstmath{CommutativeMonoid} and \lstmath{CommutativeGroup}, despite the fact that they are related. The second definition connects them, while also keeping the connection to \lstmath{CommutativeMagma} through the path from it to \lstmath{CommutativeMonoid}. 

These observations stems from the fact that we are not only interested in computing the output theory of the expression, but we are also interested in building a rich theory graph that captures as much of the structure of mathematics as possible. 

In some cases, a whole hierarchy has been developed and one may want to perform a pushout of the whole graph along a morphism, in a similar way to~\cite{cicm2019diagrams}. We encountered this situation while creating \lstmath{Semiring}, as that's when the additive and multiplicative variants of the theories are combined together. We have not implemented diagram combinators in the tog framework and leave this as future work. 
%\begin{comment}
\begin{figure}[h]
\adjustbox{scale=0.7}{   
\begin{tikzcd}
& \cn{AddMagma} \arrow[rd, hook] &  & \cn{LeftRingoid} \arrow[rd, hook]  &  \\
\cn{Magma} \arrow[dd, hook] \arrow[ru, maps to] \arrow[rd, maps to] &  & \cn{RingoidSig} \arrow[ru, hook] \arrow[rd, hook] & & \cn{Ringoid} \arrow[dd, dotted, hook] \\
 & \cn{MultMagma} \arrow[ru, hook] & & \cn{RightRingoid} \arrow[ru, hook] & \\
\cn{PointedMagma} \arrow[d, hook] \arrow[rrrr, dotted, hook] & & & & \cn{PointedMagma@Ringoid} \\
\cdots &  &  & &   \\   
 \end{tikzcd} }
\caption{Shift the \cn{PointedMagma} hierarchy to \cn{Ringoid}}
\label{fig:shiftPMtoRingoid}
\end{figure}

Another line of future work is to support general morphisms as described in Section~\ref{sec:generalmorph} and their usage in the \lstmath{mixin} combinator as described in~\cite{carette2018building}. 
\ednote{@JC: Can you please check this section more thoroughly. I am not sure I fully understand mixins}
The \lstmath{mixin} combinator computes pushout of an embedding along a general morphism. Given a general morphism \lstmath{$[\text{u}_\Delta]: \Gamma \to \Delta$} and an embedding \lstmath{$[\text{u}_\Phi] : \Gamma \to \Phi$} and two injective renaming functions 
\lstmath{$\pi_\Delta : \mid\Delta\mid \to \mathbb{V}$}
and \lstmath{$\pi_\Phi : \mid\Phi\mid \to \mathbb{V}$}, the mixin is defined as follows 
\[
\mixfun\left( u_{\Delta}, u_{\Phi}, \pi_{\Delta}, \pi_{\Phi}\right) \define
\left\{\begin{aligned}
  \mathtt{pres} & = \combineResult_1\rtimes\combineResult_2 \\
  \mathtt{embed}_{\Delta} & = \left[v_{\Delta}\right] : \Delta\rightarrow\combineResult\\
  \mathtt{view}_{\Phi} & = \left[v_{\Phi}\right] : \Phi\rightarrow\combineResult\\
  \mathtt{diag} & = \left[uv\right]:\Gamma\rightarrow\combineResult\\
\end{aligned}\right\}
\]
%  \mathtt{mediate} & = \lambda\ w_{\Delta}\ w_{\Phi}\ .\  w_{\combineResult}
where $\Xi_1 = \pi_\Delta \cdot \Delta$ is the theory presentation resulting from applying the rename function $\pi_\Delta$ to $\Delta$ via substitution. 
$\Xi_2 = \pi_\Phi \cdot \Phi^+$ is not a well-formed theory presentation, instead, it is the result of applying $\pi_\Phi$ to declarations of $\Phi$ that are not mappings of declarations in $\Gamma$. 
In~\cite{carette2018building}, proofs that the mixin operation as described above is always defined has been presented. 

For example, the morphism \lstmath{flip : Magma $\;\to\;$ FlippedMagma} shown in Section~\ref{sec:generalmorph} can be used to construct flipped \lstmath{Semigroup} as follows. 
\begin{togcode} 
FlippedSemigroup = mixin flip {} Semigroup {} 
\end{togcode} 
In this case, $\text{u}_\Delta$ is the \lstmath{flip} morphism, $\text{u}_\Phi$ is the morphism \lstmath{Magma $\;\to\;$ Semigroup}. Therefore, the resulting presentation \lstmath{pres} will have definitions from \lstmath{FlippedMagma} and the associativity axiom from \lstmath{Semigroup}. 

%\end{comment}
%  & AddMagma   &  & LeftRingoid  &  \\

% Magma &  & RingoidSig  & & Ringoid  \\
 
% PointedMagma  & MultMagma & & RightRingoid & \\
%MultMon & Zero &  & & PointedMagma@Ringoid         \\   
%
\begin{comment}
There are two different ways by which a user or a library builder can define a new theory; either by stating all its components or by reusing existing theories. While an end user might in some cases prefer the first approach for the formalization tasks, a library is more rich in information if it deploys the second approach. For example, defining a \group to be a \monoid with inverse gives us more information than what are its declarations, it also tell us about how it is related to \monoid. Most systems provide users with at least inclusions which enable them to include a verbatim version of one theory into the other, so the relation between \group and \monoid mentioned here can be captured. But, is this enough? 

%Combinators are mainly used to create connections between the new theory and existing ones. The have been around since, at least the work of Goguen and Brustall on CLEAR under the name \emph{theory building operations} \cite{Goguen1980}. Their work on theories and morphisms can be seen as an early envision of a theory graph. CLEAR has introduced two combinators to add new declarations to a theory, \lstmath|extend-signature-morphism| adds new symbols to the language of the theory and \lstmath|enrich| which adds new symbols or equations \ednote{corresponding to conservative and non-conservative extension.}The operation \lstmath|combine(T,T')| corresponds to the coproduct of the two theories. Notice how the input to the operation is two theories, not morphisms. The application of an argument to a parameterized theory is computed using the \lstmath{apply(F,$\langle$ F$_1$$\cdots$F$_n$ $\rangle$)}. This operation is arrow-based, but we were not able to find its implementation. It also has a combinator \lstmath{derive(T,$\sigma$,T')} that calculates the quotient of \lstmath{T} by \lstmath{$\sigma$}\ednote{I don't understand this one}.  
\end{comment}
\chapter{Related Work}
\label{ch:relatedwork}
\ednote{Another thing to look at, simplification and rewrite tactics}  
\ednote{look at: Generic Derivation of Induction for Impredicative Encodings in Cedille \url{http://homepage.cs.uiowa.edu/~astump/papers/cpp-2018.pdf}}
\ednote{various works have mentioned that to formalize informal math, one need to work on providing definitions for things that are obvious from an informal point of view. One of these is capretta's universal algebra in type theory that try to close this gap by translating definitions of set theory into type theory. Part of what we do is generate the definitions that they provide by hand}. 
\section{Formalizing the algebraic hierarchy}
The algebraic hierarchy is a main part of the libraries of theorem provers. Several efforts has been dedicated to organize them in a way that reflects their mathematical structure. 
The simplest way is to use inclusions to describe inheritance between two structures. This is used in~\cite{Geuvers2002} where algebraic structures are presented as dependent records and user provided coercions are used to guide the unification algorithm to infer the connections between them. As has been noted by the authors, this technique does not support multiple inheritance, so there is no way to describe that a ring is both a monoid and an abelian group. \cite{Geuvers2002} develops a hierarchy using this technique and uses it to prove the fundamental theorem of algebra. The hierarchy developed is not very deep. 

Various techniques have been developed to incorporate multiple inheritance. Canonical structures~\cite{canonical2013} is a mechanism for programming the type inference, originally introduced to handle overloading of symbols. It has been used to enable multiple inheritance in the development of the mathematical components library~\cite{mathCompLib} which has been used in the proof of the odd order (Feit-Thompson) theorem. 
% Canonical mul_monoid := Monoid.Law mulrA mul1r mulr1. 
% 
Another approach to building the algebraic hierarchy in Coq is using packed classes~\cite{Gonthier2009} which mainly solves the problem of multiple inheritance. This approach has been extended in~\cite{cohen2020hierarchy} and~\cite{sakaguchi2020validating} to overcome the complexity\ednote{read more about why it is complex} of using it to build and maintain the hierarchy. \cite{cohen2020hierarchy} creates an ELPI~\cite{elpi,elpiForCoq} plugin to Coq introducing a language for building the algebraic hierarchy whose expressions are elaborated into packed classes. One of the merits of this language is that the hierarchy can change without breaking users' code, i.e. it makes it possible to add new structures and connections between them, while keeping the older ones. \cite{sakaguchi2020validating} provides invariants and algorithms to validate the structure of the library.  

Type classes has been used to build the algebraic hierarchy in Coq and Lean. In Coq~\cite{spitters2011type}, type class $A$ extends type class $B$ by having $B$ become a field of $A$. The unification algorithm is guided by using \verb|:>| symbol instead of \verb|:| when declaring the type. Multiple inheritance is therefore possible. Lean~\cite{lean2019}, on the other hand, provide an \verb|extends| operation through which one can state all the predecessors of a class. Lean also provide \verb|attributes| that enables describing other ways in which structure connect to each other. For example, the \verb|to_additive| attribute describes that one class is the additive version of another.

These different techniques restrict the morphisms between theories to the identity. Even in the case of a rename, it is performed first and then the inclusion happens. In some cases it is not clear whether there is actually a rename morphisms between the two versions of the theory or not. On the other hand, some systems allow general morphisms, as we explain in Section~\ref{sec:background:morphisms}, which are capable of describing more complex relations between theories. Many specification systems~\cite{Goguen1980, CoFI:2004:CASL-RM, Smith99, duran2007maude} allow user provided general morphisms. They mostly refer to them as \emph{views}. It is common for these systems to provide combinators to build new theories by reusing older ones. General morphisms In the theorem proving world, Isabelle provides locale interpretations~\cite{localeIntepretations2006}, IMPS provides theory interpretations~\cite{farmer1993imps}, and MMT provides morphisms~\cite{rabe2013scalable}. Neither IMPS nor MMT provides combinators, which makes it hard to build large libraries, like the one we develop in this work, as the library developer needs to provide all theories and morphisms manually. Isabelle provide locale expressions~\cite{ballarin2003locales}, which are combinators to build locales and locale intepretations. However its \lstmath{combine} operator is based on same-name-same-thing principle, which has limitations that we discuss in~\cite{carette2018building}\ednote{probably somewhere in the thesis too, refer to the section}. 

The approach introduced in~\cite{carette2018building} allows general morphisms. The \lstmath{mixin} operation allows reusing them in a way that makes all morphisms in a theory graph commutes. Our implementation of the approach does not support general morphisms, as we indicate in Section~\ednote{put the ref}. It still allows describing names-to-names mappings, via \lstmath{rename}, which makes it more powerful than the ones that only support identity morphisms. 


%, packed classes are suggested to solve the multiple inheritance and the packaging problems\ednote{check the pakaging one}. They are used to define the algebraic hierarchy as part of the proof for Thompson-Feit theorem. Despite this improvement, packed classes need Both~\cite{cohen2020hierarchy} and~\cite{sakaguchi2020validating} criticize packed classes for the complexity of using them to build a hierarchy, as well as the complexity of maintaining a hierarchy build with them. 

%Canonical structures are used to build the algebraic hierarchy as part of the odd order theorem proof~\cite{canonical2013}. The approach is to provide unification hints for the type inference algorithms that makes it build the connection between abstract theories, or between abstract theories and their instances. 

%In~\cite{Gonthier2009} an algebraic hierarchy is developed as part of the project to formalize the proof of the Thompson-Feit theorem. Unsatisfied with telescopes and canonical structures, the author suggests packed classes to solve the problems of multiple inheritance and the coercion chain problem. The work is criticised by \cite{cohen2020hierarchy} for being complex, needs to be a coq expert to use it, and for not being robust to small changes in the hierarchy like decomposing a structure into two simpler ones. \cite{cohen2020hierarchy} builds on packed classes by providing an ELPI plugin to coq to build and maintain the algebraic hierarchy called hierarchy builder $\mathcal{HB}$. The declarations of algebraic structures written in $\mathcal{HB}$ is then elaborated into packed classes. Also~\cite{sakaguchi2020validating} also says that using packed classes needs expertize in Coq and its hard maintenance. 

%Another algebraic hierarchy is developed in~\cite{Geuvers2002} as part of the proof for the fundamental theorem of algebra. This hierarchy favors the use of dependent records, depending on coercions to guide the unification algorithm of Coq to infer connections between structures. 
%The inferred connections are mostly inclusions, what MMT would call implicit morphisms. Using only dependent records, the paper admits it is not possible to represent multiple inheritance, like the case that a ring has two monoids, one additive and one multiplicative. 

%In~\cite{spitters2011type}, type classes are used to represent algebraic structures. Inheritance is described as record fields, so a semiring would have two fields for monoids to describe that it contains two monoids. A special symbol $:>$ is used to guide the instance resolution algorithm, building the hierarchy. 

% --- lean --> type classes 

%The lean mathematical library is described in~\cite{lean2019}. The library focuses on classical mathematics, with an algebra library containing $2794$ declarations. The library uses type classes to capture the structure of mathematics. By providing type class constraints, the diamond structure can be presented. The choices for bundling is similar to ours, the carrier is a parameter to the type class, all other declarations are bundles. For morphisms, the source and target are arguments, with all other declarations being bundled, mainly to provide a better descriptive definition of composition. 

\begin{comment}
% --- Mizar --> attributed types 
\cite{Grabowski2020} agrees with us that current formalizations of the algebraic hierarchy are not general enough to serve multiple projects, and that they are usually created with specific formalization in mind. It presents the mizar approach to building the algebraic hierarchy in a way that captures their mathematical structure and facilitates the transfer of results between them. In mizar, a signature is defined as a structure. An example taken from~\cite{schwarzweller2007mizar} 
\begin{lstlisting}
definition 
struct (ZeroStr) LoopStr
(# carrier -> set,
add -> BinOp of the carrier,
Zero -> Element of the carrier #);
end;
\end{lstlisting}
Multiple inheritance between structures (signatures) is possible. \cite{Grabowski2020} gives the signature of rings and fields as follows 
\begin{lstlisting}
definition struct (addLoopStr, multLoopStr_0) doubleLoopStr
(# carrier -> set,
addF -> BinOp of the carrier,
multF -> BinOp of the carrier,
OneF -> Element of the carrier, 
ZeroF -> Element of the carrier
#);
end;
\end{lstlisting}
It is not clear how this inheritance works, as all declarations are repeated in the body of the struct. 
Axioms (properties) are added as attributes, as follows~\cite{schwarzweller2007mizar}
\begin{lstlisting}[mathescape]
definition 
let L be non empty LoopStr;
attr L is add-associative means
for x,y,z being Element of L holds (x + y) + z = x + (y + z);
$\cdots$
end;  
\end{lstlisting}
An algebraic structure is defined as an existential registration 
\begin{lstlisting}
definition 
mode Group is add-associative right_zeroed right_complementable
(non empty LoopStr);
end;
\end{lstlisting} 
\end{comment}


\section{Generative Approaches in Theorem Provers}

\subsection{Generating Definitions}
In Coq, given an inductive type, the induction principle is generated in $4$ flavors. Equality can also be generated using \verb|Scheme Equality| command. The Coq approach to generating both is criticized in~\cite{coqDeriveEquality2019} and~\cite{coqDeriveSubterm2020}. The generated induction principles are weak, especially if the type has containers.  In~\cite{coqDeriveEquality2019} equality tests and proofs are generated for inductive types using Coq-ELPI plugin. In~\cite{coqDeriveSubterm2020}, MetaCoq is used to define equality and subterm relations.\ednote{also talk about the report: Adding support for induction in Dedukti and Automating Inversion of Inductive Predicates in Coq} 

\ednote{check this paper: A Certified Multi-prover Verification Condition Generator}

\subsection{Translation / Exporting Between Systems}
Both Coq~\cite{CoqCodegen2003, cruz2003program} and Isabelle~\cite{IsabelleCodegen2010} provides code extraction mechanisms from their theories and proofs into functional programs. They face problems like how to deal with logical parts, we are not facing these problems because of the nature of what we export and because we export to theorem provers, no programming languages yet. 

Several translations between libraries of formal proofs has been done~\cite{impsToOmdoc2018, mizarToIsabelle2018, iancu2013mizar}. translating proof terms~\cite{kaliszyk2019DeclProofTerms}, and between a theorem prover and a computer algebra system~\cite{Lewis_2017}. 
The work in~\cite{Muller2017alignment} share our motivation of contributing to building large libraries of mathematics. The idea is to provide concept alignment between different theorem provers. We can see this approach useful as we expand our exporter to support different systems with different underlying foundations.  


\section{Reflection Mechanisms in Theorem Provers}
A meta programming framwork for formal verification (automation in lean)




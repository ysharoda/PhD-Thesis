\chapter{Background}

\section{Axiomatic Theory}

An axiomatic theory is a pair \lstmath{(L,$\Sigma$)}, where \lstmath{L} describes the language of the theory. It consists of \lstmath{(S,F)}. \lstmath{S} is a set of sorts and \lstmath{F} is a set of function symbols. \lstmath{$\Sigma$} is a set of axioms describing properties of objects in the language. A theory is closed under logical consequences, therefore we speak of a \emph{theory presentation}, in which only necessary axioms are given. Theory presentations are defined in a logic that provides the inference rules used to derive useful theorems.  

In a dependently typed setup, the distinction between sorts, functions symbols, and axioms need not exist. Instead a theory is seen as a telescope~\cite{de1991telescopic}. Every declaration \lstmath{$c : t$} within the telescope is defined within a context \lstmath{$\Gamma$}. This is described as \lstmath{$\Gamma \vdash c : t$}.  

A theory presentation is well-typed if every declaration \lstmath{$c : t$} is well-typed given its context. The formation rules for theory presentations are given in Figure~\ref{fig:ctx}, where $\syms{\Gamma}$ refers to the list of symbols defined in the context $\Gamma$. 
$$ \syms{\context{\varnothing}} = \EmptyThy \qquad
\syms{\context{\Gamma}\ ;\ x : \sigma} = \syms{\context{\Gamma}} \cup \left\{ x \right\}
$$
\begin{figure}[ht]
    \begin{proofrules}
        \[ \ \justifies \varnothing \ \wfctx \]
        \[ \context{\Gamma}\ \wfctx \qquad \sigma \notin \syms{\context{\Gamma}}
        \qquad \context{\Gamma} \vdash \kappa : \Box \justifies
        (\context{\Gamma}\ ;\ \sigma : \kappa)\ \wfctx \]
        \[ \context{\Gamma}\ \wfctx \qquad x\notin \syms{\context{\Gamma}}
        \qquad \context{\Gamma} \vdash \sigma : \kappa : \Box \justifies
        (\context{\Gamma}\ ;\ x : \sigma)\ \wfctx \]
    \end{proofrules}
    \caption{Formation rules for contexts as introduced in~\cite{carette2018building}}
    \label{fig:ctx}
\end{figure}

\section{Theory Morphisms}


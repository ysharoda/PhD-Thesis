\chapter{Background}
\todo{JC: Have an overview telling us: 1. what background information we’re going to need 2. and where each of those ideas will in fact be used. [If they are not, then don’t put them in!]}

\section{Axiomatic theory}
A logic is defined in~\cite{Gries1993FormalLogic} to be a set of rules defined in terms of 
\begin{itemize}
\item a set of symbols, like \lstmath{=}, \lstmath{$\wedge$}, \lstmath{$\vee$}, constants, like \lstmath{true} and \lstmath{false}, and variables, like \lstmath{p} and \lstmath{q}. 
\item a set of formulas constructed from the symbols.  
\item a set of axioms stating the properties of the symbols  
\item a set of inference rules used to derive new formulas.  
\end{itemize}
%The set of formulas is called the language of the logic. The language is defined syntactically; there is no notion of meaning or semantics in a  logic per se. 
The language of the logic provides the meta-primitives that theories in that logic uses to describe their own languages. 
An axiomatic theory \lstmath{T} in some logic is defined as the tuple \lstmath{(L,$\Gamma$)} such that 
\begin{itemize}
\item \lstmath{L} is the language of the theory consisting of sorts and function symbols. 
\item \lstmath{$\Gamma$} is the set of formulas that holds in \lstmath{T}. 
\end{itemize}
The set \lstmath{$\Gamma$} is closed under logical consequence and can be infinite in some cases. A \emph{theory presentation} of some theory \lstmath{T} is the tuple \lstmath{(L,A)} where the language is the same as \lstmath{T}, but the set of formulas \lstmath{A} is subset of \lstmath{$\Gamma$}. Note that the same theory can have different theory presentations. In this work, as is traditionally the case, we use the term theory to refer to theory presentations. 

A theory in equational logic, like the ones we deal with in this work, restricts the set \lstmath{$\Gamma$} to be a set of equations of the form \lstmath{t$_1$ = t$_2$}. We refer to this set as \lstmath{E}. 
This exposes the question of which equality is to  be used~\cite{oneThingSame2008, equalityInTPs2015}. In many cases the underlying logic offers its own propositional equality, like in tog\todo{check at this point the reader is aware of tog}. In some other cases, the equality is defined by the language of the theory as is the case when using setoids. We discuss the equality issue more in~\todo{Do we need to discuss it? Where?}.

Most algebra text books would refer to the language of the theory presentation as the signature and would separate its two components, writing it as \lstmath{(S,F)}. They often do not use the term \lstmath{theory}, but refer to sigantures with specific properties that are satisfied by the algebras. The use of axiomatic theories as we define them here is, to the best of our knowledge, tied to using them in computerized systems as in Clear~\cite{Goguen1980}. 

With dependent types and curry-howard correspondance in place, the distinction between the three components, sorts, function symbols, and axioms, is not needed anymore. Instead a theory is seen as a telescope~\cite{de1991telescopic}. Every declaration \lstmath{$c : t$} within the telescope is defined within a context \lstmath{$\Gamma$}. This is described as \lstmath{$\Gamma \vdash c : t$}.  

A theory presentation is well-typed if every declaration \lstmath{$c : t$} is well-typed given its context. The formation rules for theory presentations are given in Figure~\ref{fig:ctx}, where $\syms{\Gamma}$ refers to the list of symbols defined in the context $\Gamma$. 
$$ \syms{\context{\varnothing}} = \EmptyThy \qquad
\syms{\context{\Gamma}\ ;\ x : \sigma} = \syms{\context{\Gamma}} \cup \left\{ x \right\}
$$
\begin{figure}[ht]
    \begin{proofrules}
        \[ \ \justifies \varnothing \ \wfctx \]
        \[ \context{\Gamma}\ \wfctx \qquad \sigma \notin \syms{\context{\Gamma}}
        \qquad \context{\Gamma} \vdash \kappa : \Box \justifies
        (\context{\Gamma}\ ;\ \sigma : \kappa)\ \wfctx \]
        \[ \context{\Gamma}\ \wfctx \qquad x\notin \syms{\context{\Gamma}}
        \qquad \context{\Gamma} \vdash \sigma : \kappa : \Box \justifies
        (\context{\Gamma}\ ;\ x : \sigma)\ \wfctx \]
    \end{proofrules}
    \caption{Formation rules for contexts as introduced in~\cite{carette2018building}}
    \label{fig:ctx}
\end{figure}

\section{Theory Morphisms}
\label{sec:background:morphisms}
\todo{If T1 |= phi then T2 |= view-from-T1-to-T2(phi)}
Morphisms are used to capture the structure of mathematics, by describing how theories are related to each other. In mathematical texts, a theorem proved for an arbitrary \lstmath{Monoid} would be used when considering an arbitrary \lstmath{Group} without extra work. In formal mathematics, this can only be done if a morphism between \lstmath{Monoid} and \lstmath{Group} exists. The morphism specifies how results in \lstmath{Monoid} can be interpreted in \lstmath{Group}. 

A morphism $\arrow{[v]}{\Gamma}{\Delta}$ consist of a list of assignments $[v]$, a source theory \lstmath{$\Gamma$}, and a target theory \lstmath{$\Delta$}. $[v]$ assigns to every symbol in $\Gamma$ an expression of $\Delta$. A term \lstmath{t} in the language of $\Gamma$ can be translated into a term \lstmath{t$^\prime$} in the language of $\Delta$ using substitution, such that  \lstmath{t$^\prime$ = t$[v]$}. 

The formation rules for views, as given in~\cite{carette2018building} is given in Figure~\ref{fig:views}. 
\begin{figure}[ht]
    \begin{proofrules}
        \[ \context{\Delta}\ \wfctx \justifies \view{}{\varnothing}{\Delta} \]
        \[ (\context{\Gamma}\ ;\ x : \sigma)\ \wfctx \qquad
        \view{v}{\Gamma}{\Delta} \qquad
        \context{\Delta} \vdash r : \substitution{\sigma}{v}{} \justifies
        \view{v,x \mapsto r}{(\context{\Gamma}\ ;\ x : \sigma)}{\context{\Delta}} \]
    \end{proofrules}
    \caption{Formation rules for morphisms.}
    \label{fig:views}
\end{figure}

It is worth mentioning that the mapping is only a part of the morphism. A morphism consists of the source and destination theories as well as the mapping. 

Connecting theories have been known for a long time in logic~\cite{tarski1953undecidable, enderton1972mathematical} under the name \emph{theory interpretations}. The same name is used by IMPS~\cite{farmer1993imps, InterpIMPS1994}. CLEAR~\cite{Goguen1980}, OBJ and their successors used the term \emph{morphisms}, maybe because of using category theory for semantics. The term \emph{view} has also been used to refer to the same concept by Maude, MathScheme, and MMT. 

We distinguish between three type of morphisms 

\subsection{Identity Morphism}
\label{sec:idmorph}
If $\arrow{[v]}{\Gamma}{\Delta}$ is an identity morphism, then $[v]$ maps every symbol $x \in \syms{\Gamma}$ to itself. This implies that $[v]$ is a bijection and that $x[v] = x$. 

Identity morphisms exist between two theories if the source is included verbatim in the destination, like in the case when \lstmath{Group} is defined as an extension of \lstmath{Monoid}. It is the simplest form of morphisms and allow transport of results without the need to perform substitution 

\subsection{Embedding}
\label{sec:embedding}
If $\arrow{[v]}{\Gamma}{\Delta}$ is an embedding, then $[v]$ is a bijection that maps every symbol $x \in \syms{\Gamma}$ to a symbol $r \in \syms{\Delta}$, which is not necessarily itself. 
\ednote{introduce the notation that an embedding morphism $m$ is represented using $\tilde{m}$. Since the identity is an embedding it is also represented as $\tilde{id}$}
\ednote{present the $\vars$ and that \lstmath{embed : $\vars \to \vars$}}

Consider for example, the morphism from \verb|Magma| to \verb|AdditiveMagma|
\begin{equation*}\label{eq:additiveview}
\begin{tikzpicture}[node distance=9.0cm, auto,baseline=(current bounding box.center)]
\node (P) {$
    \begin{thyex}
    \thyrow{A}{\tmop{Type}}
    \thyrow{op}{A \rightarrow A \rightarrow A}
%    \thyrow{assoc_op}{\cdots}
    \end{thyex} $};
\node (B) [right of=P] {$
    \begin{thyex}
    \thyrow{A}{\tmop{Type}}
    \thyrow{+}{A \rightarrow A \rightarrow A}
%    \thyrow{assoc_+}{\cdots}
    \end{thyex} $};
\draw[->] (P) to node {$[A \mapsto\ A, 
    op \mapsto\ + ]$} (B);
\end{tikzpicture}
\end{equation*}
A term $t \in \Gamma$ is transported to $\Delta$ as $t[v]$, i.e.: by applying the substitution $[v]$ to the term $t$. 

\subsection{General Morphism}
\label{sec:generalmorph}
A morphism $\arrow{[v]}{\Gamma}{\Delta}$ is a general morphism if it maps symbols $x \in \syms{\Gamma}$ to terms $r \in \Delta$. 

An example is a morphism that flips a binary operation, i.e.: maps \lstmath{op x y} to \lstmath{op y x}
\begin{equation}\label{eq:flipmagmaview}
\begin{tikzpicture}[node distance=9.0cm, auto,baseline=(current bounding box.center)]
\node (P) {$
    \begin{thyex}
    \thyrow{A}{\tmop{Type}}
    \thyrow{op}{A \rightarrow A \rightarrow A}
    \end{thyex} $};
\node (B) [right of=P] {$
    \begin{thyex}
    \thyrow{A}{\tmop{Type}}
    \thyrow{op}{A \rightarrow A \rightarrow A}
    \end{thyex} $};
\draw[->] (P) to node {$[A \mapsto\ A,
    op \mapsto\ \mathsf{flip}\ op]$} (B);
\end{tikzpicture}
\end{equation}

\section{Theory Graph}\label{sec:background:theorygraph}
One way to organize theories is using theory graphs. They are needed when dealing with large libraries~\cite{kohlhase2010towards}. A theory graph is an acyclic directed graph consisting of a set of theories, as nodes, and morphisms, as edges between them. 

In systems that are based on categorical semantics, a theory graph is seen as a diagram in the category of theories and theory morphisms. Specware uses the keyword \emph{diagram} to build them. The work in~\cite{developmentGraph2000}, based on CASL, refer to them as \emph{development graphs}. 

\ednote{talk about Hets, being a graph of different logics, based on  institutions}

\section{Little Theories}
The little theories approach is introduced in~\cite{LittleTheories}. It ensures that if a statement \lstmath{s} is proven in context \lstmath{$\Gamma$}, then every statement in \lstmath{$\Gamma$} is required to prove \lstmath{s}. In this case, we say \lstmath{$\Gamma$} is the \emph{minimal axiomatization} needed to prove \lstmath{s}. This implies that theorems are proved in different contexts based on the amount of structure needed to prove them. In contrast, a big theory approach would use only one big theory for proving all results. 

Using little theories maximizes reuse of results. Results proven in the theory \lstmath{$\Gamma$} can be transported to all theories \lstmath{$\Delta$} whenever a morphism $m : \Gamma \to \Delta$ exists. This emphasizes the role of morphisms for increasing usability and reducing redundancy when dealing with formal systems.  

\section{Tiny Theories}
Tiny theories is a refinement of little theories, in which only one new piece of information is added at a time~\cite{mathscheme2011experiments}. To make this more clear, let's consider a library that has the theories \lstmath{PointedMagma} and \lstmath{Unital} defined as follows. 
\begin{lstlisting}[mathescape]
theory PointedMagma = { 
  A : Type 
  e : A 
  op : A $\to$ A $\to$ A }
theory Unital = {
  A : Type 
  e : A 
  op : A $\to$ A $\to$ A 
  lunit : $\cdots$
  runit : $\cdots$
}   
\end{lstlisting}
These definition overlook that in some cases one might want to talk about an operation with only right unit, like in the case of subtraction on integers. One will then need to add a new theory that is similar to \lstmath{Unital} without the \lstmath{lunit} declaration. Any theorems proved in the context of \lstmath{Unital} cannot be used, even if it only depends on \lstmath{lunit}. 

\section{Universal Algebra}

\section{Substitution}

\section{Category Theory}
explain pushout and how it is calculated 

\section{Multi-Staged Programming}
\label{sec:background:msp}

\section{Finally Tagless}
\label{sec:background:tagless}
\documentclass[]{article}

\usepackage{enumitem}

\begin{document}
    
\section{Introduction}
\subsection{Overview}
\begin{itemize}
    \item Large Math libraries are useful. evidences are: the vision of QED, the paper about UDML challenge by WF, look into publications and talks of the Xena project and LML workshop for evidence 
    \item A generative approach can proof useful 
    \begin{itemize}
        \item the components of an algebra library 
        \item why it is possible and useful to generate them 
    \end{itemize}
\end{itemize}

\subsection{Research Problem}
\begin{itemize}
    \item Theory presentations are written in a formal language that has a specific structure 
    \item They look different in different systems because they are loaded with design decisions 
    \item Accordingly, related constructions look different and are repeated for every theory 
    \item Mathematically, there is a uniform way to look at them described by universal algbera 
\end{itemize}    
    
\subsection{Contributions}
\begin{itemize}
    \item Highlighting redundancy in current library building practice 
    \item Identifying X constructions that can be generated for uni-sorted FOL theories 
    \item Providing an extensible framework for implementing them 
    \item Implementing a library of algebraic theories that emphasizes the role of arrows 
    \item Exporting the generated constructions to Agda 
\end{itemize}    

\subsection{Broader Scope}
\begin{itemize}
    \item we are working towards library organized as a theory graph of biform theories at the center of a tetrapodal structure.  
\end{itemize}

\section{Background}
\begin{itemize}
    \item Little Theories 
    \item Tiny Theories 
    \item Categories of theory presentations / contexts / extensions 
    \item Axiomatic theories and formal languages 
\end{itemize}
    
\section{Design of our framework}
\begin{itemize}
    \item specification of the different pieces of the interpreter (flattener, generator, and exporter). 
    \item benefits of having this framework: better usability, maintainability, and less time-to-market. 
\end{itemize}    

\section{Tog}
\begin{itemize}
    \item a short section introducing the basics of Tog 
\end{itemize}

\section{Library of Algebra}
\subsection{Overview}
    \begin{itemize}
        \item Why the MathScheme library maximizes reuse: focusing on diamonds and arrows in MathScheme versus in other systems 
    \end{itemize}
\subsection{Combinators}
    \begin{itemize}
        \item Explain the combinators: syntax, operational semantics, and categorical semantics (briefly)
    \end{itemize}
\subsection{Implementation}
    \begin{itemize}
        \item Design decisions: 1. combine-over instead of combine, 2. not using mixin 
        \item Graph building: implementation og combinators 
    \end{itemize}
\subsection{Interesting Cases}
PointedMagma, Monoid, and Semiring 

\section{Constructions For Free!}    
\begin{itemize}
    \item Infrastructure
    \begin{itemize}
        \item Functions, Datatypes, and Records: definitions and applications 
    \end{itemize}
    \item Constructions 
    \begin{itemize}
        \item for each: motivation, definition from universal algbera, and implementation. 
    \end{itemize}
\end{itemize}

\section{Exporting}

\section{Related Work}

\section{Discussion}
Generalization to multisorted and higher types. 

\section{Conclusion}

    
\end{document}
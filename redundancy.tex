\begin{comment} 
Dr. Carette's remarks: 
“formal systems community” – I have rarely heard it called that. You might want to be more precise. You only cover ITPs that use logics more expressive than FOL.
I think you’ll need to justify (and least via examples) your first sentence, if you’re going to include it at all. I would seriously consider deleting it.
3.2 focuses way too much on stuff that I’ve done. You need to diversify more!
I was expecting some actual statistics to show up here, not just examples! 
this chapter, as written has too little actual content in it. I’m pretty sure you’ve looked much deeper and harder into those libraries. I think reporting on that could be a solid, tangible contribution.
\end{comment} 

\chapter[Boilerplate In Libraries]{Boilerplate In Libraries\footnote{This chapter is adapted from~\cite{leverageCICM2020}.}}
\label{ch:redundancy}
%One of the observations we build our work on is that libraries of formal systems contain developer provided definitions that can be automatically generated. In this chapter we support this observation by listing 
%The formal systems community has mostly focused on making necessary things possible. Our focus is more oriented to making uniform definitions free. In this chapter we highlight the redundancy that is currently a part of the development of theories in formal systems. In Section~\ref{sec:redun:libraries} we focus our attention to libraries of formal systems, while in Section~\ref{sec:redun:user_projects} we show how some projects are redfines algebraic structure and some other constructions because they are unable to use the definitions provided by libraries of the system they are using. 
%\section{Algebra Theories in Libraries}\ednote{currently a copy-paste from the paper}
%\label{sec:redun:libraries}

One of our observations is that current formalizations of Algebra contain quite a
bit of information that is ``free'' in the sense that it can be
mechanically generated from basic definitions. For example, given a theory
X, it is mechanical to define X-homomorphisms. 

%To do this within a system is extremely difficult, as it would require introspection and for theory
%\emph{definitions} to be first-class citizens, which is not the case for any
%system based on type-theory that we are aware of\ednote{still checking idris}. Untyped systems
%in the Lisp tradition do this routinely, as does Maude~\cite{Maude}, which
%is based on \emph{rewriting logic}; the downside is that there is no
%difference between meaningful and meaningless transformations in these
%systems, only between ``runs successfully'' and ``crashes''. However,
%these constructions are fully typeable and, moreover, are not system-specific
%(as they can be phrased meta-theoretically within Universal
%Algebra), even though an implementation has to be aware of the syntactic
%details of each system.

Lest the reader think that our quest is a little quixotic, we first look at
current libraries from a variety of systems, to find concrete examples of
human-written code that could have been generated. We look at Agda,
Isabelle/HOL and Lean in particular. More specifically, we look at
\href{https://github.com/agda/agda-stdlib/releases/tag/v1.4}
{version 1.4 of the Agda standard library}
and 
\href
{https://github.com/leanprover-community/mathlib/releases/tag/snapshot-2019-10}
{2019 release of Lean's mathlib}, where we link to the proper release tag.

We use the theory \verb|Monoid| as our running example, and we 
highlight the reusable components that the systems use to make writing the
definitions easier and more robust. 

%Between those definitions provided by the developers in different libraries, the overhead of having to change them when new design decisions are taken, and the missing constructions for some of the theories in the libraries, we have tried to quantify to some degree the effort that is being done by developers that can be elevated by providing automatic generators. 

\section{Agda Standard Library}
%How do the libraries of our three systems\footnote{We do not have enough room
%    to give an introduction to each system; hopefully each system's syntax is
%    clear enough for the main ideas to come through.}
%represent homomorphism?
The Agda standard library defines the following constructions related to \lstmath{Monoid}: 
\begin{enumerate}
\item \href{https://github.com/agda/agda-stdlib/blob/4099f6184a7d8cd4c02931c3ef5a95966ab4cbb6/src/Algebra/Bundles.agda}{Raw Monoid}: The \emph{raw} representation of a theory is a definition of its signature.. \lstmath{RawMonoid} is defined in the standard library as 
\begin{agdacode} 
record RawMonoid c ~$\ell$~ : Set (suc (c ~$\sqcup$~ ~$\ell$~)) where 
  infixl 7 _~$\bullet$~_
  infix 4 _~$\approx$~_
  field 
   Carrier : Set c 
   _~$\approx$~_ : Rel Carrier ~$\ell$~ 
   _~$\bullet$~_ : Op~$_2$~ Carrier 
   ~$\varepsilon$~   : Carrier 
\end{agdacode} 
The definition of \lstmath{RawMonoid} is identical to that of \lstmath{Monoid} except for one declaration that instantiates the \lstmath{isMonoid} record that checks for the properties of a \lstmath{Monoid}. 

\item \href{https://github.com/agda/agda-stdlib/blob/c61b159363ce2390049ce8e1e5422f61f17ec3b7/src/Algebra/Solver/Monoid.agda}{Open Term Language and Evaluator}:
The ``term language'' of a theory is the (inductive) data type
that represents the syntax of well-formed terms of that theory,
along with an interpretation function from \emph{expressions} 
to the carrier of the (implicitly single-sorted) given theory, i.e.
its denotational semantics.

In Agda, the definition of \lstinline|Monoid| term language is straightforward:
\begin{agdacode}
data Expr (n : ~$\mathbb{N}$~) where 
  var : Fin n ~$\to$~ Expr n 
  id : Expr n 
  _~$\oplus$~_ : Expr n ~$\to$~ Expr n ~$\to$~ Expr n 
\end{agdacode}
Defining the interpretation function requires the concept of an environment.
An environment associates a value to every variable, and the semantics
associates a value (of type \verb|Carrier|) to each expression of \verb|Expr|.
\begin{agdacode}
Env : Set _ 
Env = ~$\lambda$~ n ~$\rightarrow$~ Vec Carrier n 

~$\llbracket$~_~$\rrbracket$~ : ~$\forall$~ {n} ~$\to$~ Expr n ~$\to$~ Env n ~$\to$~ Carrier 
~$\llbracket$~ var x ~$\rrbracket$~ ~$\upvarrho$~ = lookup ~$\upvarrho$~ x 
~$\llbracket$~ id ~$\rrbracket$~ ~$\upvarrho$~ = ~$\epsilon$~ 
~$\llbracket$~ e~$_1$~ ~$\oplus$~ e~$_2$~ ~$\rrbracket$~ ~$\upvarrho$~ = ~$\llbracket$~ e~$_1$~ ~$\rrbracket$~ ~$\upvarrho$~ ~$\cdot$~ ~$\llbracket$~ e~$_2$~ ~$\rrbracket$~ ~$\upvarrho$~ 
\end{agdacode}

These definitions are not found with the definitions of the
algebraic structures themselves, but rather as part of the
\emph{Solver} for equations over that theory.

%\item \href{https://github.com/agda/agda-stdlib/blob/35253c831b3e996cb8633c8f12b353231535dfe7/src/Algebra/Construct/Zero.agda}{Zero Monoid}
%The zero theory is the trivial theory 
\item \href{https://github.com/agda/agda-stdlib/blob/5365791e21af9abb324aa3721571bdceee919932/src/Algebra/Construct/DirectProduct.agda}{Product}:
Until recently, there was no definition of the product of algebraic
structures in the Agda library.  A 
\href{https://github.com/agda/agda-stdlib/pull/1109}{recent pull request}
has suggested adding these, along with other constructions.  The
following hand-written definition has now been added:
\begin{agdacode}
monoid : Monoid a ~$\ell_1$~ ~$\to$~ Monoid b ~$\ell_2$~ ~$\to$~ Monoid (a ~$\sqcup$~ b) (~$\ell_1$~ ~$\sqcup$~ ~$\ell_2$~)
monoid M N = record
  { ~$\varepsilon$~ = M.~$\varepsilon$~ , N.~$\varepsilon$~
  ; isMonoid = record 
     { isSemigroup = Semigroup.isSemigroup 
                     (semigroup M.semigroup N.semigroup)
     ; identity = (M.identity~$^{l}$~ , N.identity~$^{l}$~ <*>_)
                  , (M.identity~$^{r}$~ , N.identity~$^{r}$~ <*>_)
     }
  } where module M = Monoid M; module N = Monoid N 
\end{agdacode}
where \lstmath{semigroup} is the definition of the product theory of \lstmath{Semigroup}. 

\item \href{https://github.com/agda/agda-stdlib/blob/e34a31f80b215812ab26c10f84c9a658eeda3110/src/Algebra/Morphism/Structures.agda}{Morphisms}
\lstmath{Monoid} homomorphism is defined in the Agda standard library using \lstmath{Magma} homomorphism as follows: 
\begin{togcode} 
record IsMonoidHomomorphism (~$\llbracket$~_~$\rrbracket$~: A ~$\to$~ B) : Set(a ~$\sqcup$~ ~$\ell_1$~ ~$\sqcup$~ ~$\ell_2$~) where 
 field
   isMagmaHomomorphism : IsMagmaHomomorphism ~$\llbracket$~_~$\rrbracket$~
   ~$\varepsilon$~-homo   : Homomorphic~$_0$~ ~$\llbracket$~_~$\rrbracket$~ ~$\varepsilon_1$~ ~$\varepsilon_2$~
\end{togcode} 

Monomorphism and isomorphism are also provided in the library, defined in terms of homomorphisms. 
\end{enumerate}
These constructions constitute $7$ definitions spanning over $35$ lines for only the theory \lstmath{Monoid}. They are also repeated for other theories. The term language and evaluator for \lstmath{Monoid} are repeated verbatim for both theories 
\href{https://github.com/agda/agda-stdlib/blob/c61b159363ce2390049ce8e1e5422f61f17ec3b7/src/Algebra/Solver/CommutativeMonoid.agda}
{\lstinline|CommutativeMonoid|}
and 
\href{https://github.com/agda/agda-stdlib/blob/c61b159363ce2390049ce8e1e5422f61f17ec3b7/src/Algebra/Solver/IdempotentCommutativeMonoid.agda}
{\lstinline|IdempotentCommutativeMonoid|}. The \lstmath{Raw} versions are provided for $7$ theories; \lstmath{Magma}, \lstmath{Monoid}, \lstmath{NearSemiring}, \lstmath{Semiring}, \lstmath{Ring}, and \lstmath{Lattice}. The definitions of the $3$ morphisms are provided for the same theories. 

The direct product is defined for $10$ theories. From the $7$ that we defined above, only \lstmath{Magma}, \lstmath{Monoid}, and \lstmath{Group} have definitions of direct product. In addition to those $3$ theories, It is defined for \lstmath{Semigroup}, \lstmath{Band}\lstmath{CommutativeSemigroup}, \lstmath{Semilattice}, \lstmath{CommutativeMonoid}, \lstmath{IdempotentCommutativeMonoid}, and \lstmath{AbelianGroup}. Beside these definitions, the products of the signatures of \lstmath{Magma}, \lstmath{Monoid}, and \lstmath{Group} is given in the library.  

These give us a total of $47$ definitions that are provided by the library developers, but could instead be generated, bearing in mind that not all constructions are provided for all theories. Also, constructions are not provided for additive or multiplicative versions of theories like \lstmath{Monoid} and \lstmath{Group}. A generative algorithm would be able to provide those variants of the constructions, at no extra cost. 
% 2 * 3 (term langs) 
% + 7 (raw definitions) 
% + 3 * 7 (the morphisms) 
% + 10 (direct products) 
% + 3 (the raw direct products) 
%The $3$ morphisms are repreated for $6$ theories;  
%The product theories are repeated for $10$ theories 
%The raw theories are defined for $7$ theories 
%Note that for each of these theories, there are the additive and multiplicative versions which agda overlooks in its library depending on its module rename function. 

It is worth noting that the defintions in the Agda standard library employ modularity when defining structures, like the definition of \lstmath{IsMonoidHomomorphism} which depends on \lstmath{IsMagmaHomomorphism}. Raw definitions from universal algebra do not support this modularity and, therefore, the generated expressions would be more \emph{flat}, i.e. include the actual declarations instead of importing them from a different structure. Having flat definitions is, in some cases, a good way to abstract over library design. Nevertheless, we do not want to lose the connections between different theories. To solve this problem, we support a library organized as theory graph on which a flattener can be built. We leave working fully with unflattened theories as future work.  

\subsubsection*{Summary}
\begin{center}
\begin{tabular}{| c || c |}
%\multicolumn{2}{c}{\textbf{Agda Standard Library}} \\ \hline 
\hline 
\textbf{Construction} & \textbf{Number of Occurrences} \\ \hline 
Signatures & 7 \\ \hline
Homomorphisms & 7  \\ \hline
Monoomorphisms & 7 \\ \hline
Isomorphisms & 7 \\ \hline
Products & 10 \\ \hline
Products of Signatues & 3 \\ \hline
Term Language & 3 \\ \hline
Evaluation Function & 3 \\ \hline
%\multicolumn{2}{c}{\textbf{Lean's Mathlib}} \hline
\end{tabular}
\end{center}

\section{Lean MathLib}
The \href{https://github.com/leanprover-community/mathlib/blob/4bb8d4475f897c8997100d31fe84b33050444374/src/algebra/group/hom.lean}
{\lstmath{homomorphism}} 
of monoids is defined in two ways in mathlib. 
One way is the \emph{unbundled} predicate style definition in which the homomorphism function is a parameter to the class definition.  
\begin{leancode} 
class is_monoid_hom [monoid ~$\alpha$~] [monoid ~$\beta$~] (f : ~$\alpha \to \beta$~) 
   extends is_mul_hom f : Prop :=
    (map_one : f 1 = 1)
\end{leancode} 
\noindent where \lstmath{is_mul_hom} is the definition of homomorphism of multiplicative magma, which lean refers to as \lstmath{mul}. A very similar definition is provided for \lstmath{add_monoid}. 
The other is the \emph{bundled} definition in which the homomorphism function is part of the declarations of the structure, not a parameter to it. 
\begin{leancode}     
structure monoid_hom (M : Type*) (N : Type*) [monoid M] [monoid N] :=
  (to_fun : M ~$\to$~ N)
  (map_one' : to_fun 1 = 1)
  (map_mul' : ~$\forall$~ x y, to_fun (x * y) = to_fun x * to_fun y)    
\end{leancode} 
The library provide the unbundled (class) definitions for many theories, including \lstmath{group}, \lstmath{semiring}, and \lstmath{ring}. These definitions are marked deprecated. We were able to only find the bundled definitions for 
\href{https://github.com/leanprover-community/mathlib/blob/4bb8d4475f897c8997100d31fe84b33050444374/src/algebra/group/hom.lean}
{\lstmath{monoid_hom}} and 
\href{https://github.com/leanprover-community/mathlib/blob/4bb8d4475f897c8997100d31fe84b33050444374/src/algebra/ring.lean}
{\lstmath{ring_hom}}. 

The lean library also have definitions for the 
\href{https://github.com/leanprover-community/mathlib/blob/4bb8d4475f897c8997100d31fe84b33050444374/src/algebra/pi_instances.lean}{product} 
of some theories. In a hierarchy ranging from \lstmath{has_add} and \lstmath{has_mul} to \lstmath{nonzero_comm_ring}, $22$ definitions of products are defined. 
%has_add, has_mul, has_zero, has_one, has_neg, has_inv, add_semigroup, semigroup, add_monoid, monoid, add_group, group, add_comm_semigroup, comm_semigroup, add_comm_monoid, comm_monoid, add_comm_group, comm_group, semiring, ring, comm_ring, nonzero_comm_ring, 
It contains definitions of 
\href{https://github.com/leanprover-community/mathlib/blob/4bb8d4475f897c8997100d31fe84b33050444374/src/group_theory/submonoid.lean}{\lstmath{is_submonoid}},
\href{https://github.com/leanprover-community/mathlib/blob/4bb8d4475f897c8997100d31fe84b33050444374/src/group_theory/subgroup.lean}{\lstmath{is_subgroup}}, 
their additive variants, and  
\href{https://github.com/leanprover-community/mathlib/blob/4bb8d4475f897c8997100d31fe84b33050444374/src/ring_theory/subring.lean}{\lstmath{is_subring}} . %The \href
%{https://isabelle.in.tum.de/website-Isabelle2020/dist/library/HOL/HOL-Algebra/index.html} 
%{Isabelle/HOL library} also contains definitions for homomorphisms and quotient of \lstmath{monoid} and multiple other theories. 

% is_add_submonoid, is_submonoid
% is_add_subgroup, is_subgroup




\begin{comment} 
The product for \lstmath{monoid} theory is defined as 
\begin{leancode} 
instance [monoid ~$\alpha$~] [monoid ~$\beta$~] : monoid (~$\alpha \times \beta$~) :=
{ one_mul := assume a, 
       prod.rec_on a ~$\$$~ 
        ~$\lambda$~ a b, mk.inj_iff.mpr ~$\langle$~one_mul _, one_mul _ ~$\rangle$~,
  mul_one := assume a, 
       prod.rec_on a ~$\$$~ 
        ~$\lambda$~ a b, mk.inj_iff.mpr ⟨mul_one _, mul_one _⟩,
  .. prod.semigroup, .. prod.has_one }
\end{leancode} 
\end{comment} 

\begin{comment}

The mathlib library contain definitions of 
\href{https://github.com/leanprover-community/mathlib/blob/700d5761d288e0498b141818c6482f01d9c02f1a/src/algebra/group/defs.lean}{\lstmath{monoid}}
and its additive version as follows 
\begin{leancode}
class monoid (M : Type u) extends semigroup M, has_one M :=
  (one_mul : ~$\forall$~ a : M, 1 * a = a) (mul_one : ~$\forall$~ a : M, a * 1 = a)
class add_monoid (M : Type u) extends add_semigroup M, has_zero M :=
  (zero_add : ~$\forall$~ a : M, 0 + a = a) (add_zero : ~$\forall$~ a : M, a + 0 = a)
attribute [to_additive] monoid
\end{leancode}
This example shows redundancy in theory definition. The \lstmath{add_monoid} class can be obtained from \lstmath{monoid} by renaming \lstmath{1 to 0}, \lstmath{* to +}, and the names of the axioms. 
%add_comm_semigroup
%add_left_cancel_semigroup
%add_right_cancel_semigroup
%add_comm_monoid
%add_left_cancel_monoid
%add_left_cancel_comm_monoid
%add_right_cancel_monoid
%add_right_cancel_comm_monoid
%add_cancel_monoid
%add_cancel_comm_monoid
%add_group
%add_comm_group
\end{comment} 

\begin{comment} 
%\section{Algebra Theories in User Projects}
%\label{sec:redun:user_projects}

We continue investigating the redundancy in code written by developers. In the previous section we looked into general purpose libraries. We now turn our attention to user projects highlighting the cases when users had to redefine information already existing in libraries. 

The theories and data structures project~\cite{theoriesAndDts}, implemented in Agda, explores how algebraic theories give rise to some common data structures. A natural first step is to formalize the algebraic theories of interest to the project. Despite the existence of some of these algebraic structures in the standard library, like the \lstmath{Monoid} theory with the definition above, the project developers redefined many of them~\footnote{https://github.com/JacquesCarette/TheoriesAndDataStructures/tree/master/Structures}, like \lstmath{Magma, }\lstmath{Monoid}, \lstmath{CommMonoid}, and \lstmath{AbelianGroup}. In some cases, like in \lstmath{Monoid}, the definitions avoided setoids, while in others, like in \lstmath{CommMonoid} it used it. This highlights how important it is to provide a way of customizing definitions while using them. 

The MGG project~\cite{carette2011generative} deals with the computational geometry software as a family of programs and aim to define abstractions from which different software pieces can be generated. One of the layers common to many members of this software family is algebra. In \lstmath{Algebra.lhs} the finally tagless representation of many algebraic structures are given as follows 
\begin{hscode}
class Monoid repr n where
  add :: repr n -> repr n -> repr n
  zero :: repr n

class Monoid repr n => AdditiveGroup repr n where
  neg :: repr n -> repr n
  sub :: repr n -> repr n -> repr n
  int_pow :: Integer -> repr n -> repr n
  of_int :: Integer -> repr n
\end{hscode}
We show in Chapter~\ref{ch:generation} that the finally tagless representation can be generated. Another form of redundancy, is defining the \lstmath{Monoid} instance of various number representations, like float, integer and rational, along with the code and staged versions. For example, in Float.lhs, we find the following definition 
\begin{hscode}
instance Monoid BaseImm Double where
  add = liftA2 (+)
  zero = pure (0.0)

instance Monoid Code Double where
  add (Code x) (Code y) = Code [| $(x) + $(y) |]
  zero = Code [| 0.0 |]

instance Monoid Staged Double where
  add a b = monoid Alg.zero add add a b
  zero = of_atom (0.0)
\end{hscode}
with instances of the three version of \lstmath{Double} for \lstmath{Field}, \lstmath{AdditiveGroup}, \lstmath{Norm}, and \lstmath{Ring}. The same definitions in repeated for \lstmath{Integer} type in Integer.lhs file. 
\begin{hscode}
instance Monoid BaseImm Integer where
  add = liftA2 (+)
  zero = pure 0

instance Monoid Code Integer where
  add (Code x) (Code y) = Code [| $(x) + $(y) |]
  zero = Code [| 0 |]

instance Monoid Staged Integer where
  zero = of_atom $ 0
  add = monoid A.zero add add
\end{hscode}
and for rationals in Rational.lhs\ednote{Problem with this example is the repo has limited accessibility.}

\end{comment} 

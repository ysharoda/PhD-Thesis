\prefacesection{Abstract}

Building a large library of mathematical knowledge is a complex and labour intensive task.
By examining current libraries of mathematics, we see that the human effort put in building them is not entirely directed towards tasks that need human creativity. Instead, a non-trivial amount is directed towards redefining basic mathematical knowledge. 
%Instead of focusing all the human effort on formalizing more knowledge, we find that some basic mathematical concepts are being redefined again and again based on changing design decisions. 

In this work, we propose a generative approach to library building, where definitions that are normally provided by library developers are automatically generated by a meta-program. We focus our attention to algebra libraries. Algebraic theories are highly structured and their commonalities has been well-studied in universal algebra. We use theory presentation combinators to capture the structure of algebraic theories. Universal algebra definitions are used to derive definitions of constructions, like homomorphism and term languages, from algebraic theory presentations. The result is an interpreter that, given $227$ theory expressions, builds a library of over $5000$ definitions.


